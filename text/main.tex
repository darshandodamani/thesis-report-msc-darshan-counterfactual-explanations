\begin{frontmatter}
    \begin{abstract}
   Deep generative models, such as Variational Autoencoders (VAEs), are widely used in autonomous driving for learning compact latent representations of input data. However, like other deep neural networks, VAEs are often regarded as black-box models, making it difficult to interpret why a particular decision was made and which attributes contributed to the prediction. This lack of transparency raises concerns about trust and reliability, particularly in safety-critical applications such as autonomous vehicle decision-making. 
   
   One approach to imprioving model interpretability is through counterfactual explanations, which aim to determine the minimal changes required to alter the model's prediction. Counterfactual Explanations provide human-interpretable insights by identifying which features are most influential in a decision, offering an alternative to traditional feature attribution methods. In this work, I propose a novel introspection technique that leverages a deep generative model (VAE) combined with feature masking strategies to generate counterfactual explanations for black-box classifiers. Our approach modifies input images and latent space representations to determine what meaningful changes would lead to a different classification outcome.
   
   To demonstrate the effectiveness of this approach, experiments are conducted using the CARLA dataset, a widely used simulation environment for autonomous driving research. The proposed method allows for a deeper understanding of classifier decisions by revealing how slight modifications in the input or latent representation affect predictions. This work contributes to the broader goal of enhancing transparency and explainability in deep learning models for autonomous systems, addressing the increasing demand for trustworthy AI in real-world applications.
    \end{abstract}

       
    
    \tableofcontents
    % \chapter*{Acknowledgements} % optional
    % I thank my cat Bubbles!
    % \listoffigures % optional
     \listoftables % optional

         % \begin{symbols}
    %     \symbolentry{$\alpha$}{Learning rate, controls the step size during optimization}
    %     \symbolentry{$\gamma$}{Discount factor in reinforcement learning}
    %     \symbolentry{$\theta$}{Model parameters in a neural network}
    %     \symbolentry{$\pi$}{Policy in reinforcement learning}
    %     % Add more symbols as needed
    % \end{symbols}
    
\end{frontmatter}

\chapter{Introduction} \label{Introduction}
\section{Background and Motivation}
Deep neural networks (DNNs) have demonstrated superior performance in solving complex problems in numerous domains, including image classification~\autocite{s22239544}, medical diagnosis~\autocite{SHAMSHIRBAND2021103627}, natural language processing~\autocite{Wang23}, and autonomous driving~\autocite{s19092064}. However, their inherent complexity often makes them difficult to interpret, leading to concerns about trust and safety.

When humans and machines use complex machine learning (ML) models to make crucial decisions, a vital concern is whether humans can trust the model and its predictions. This requirement is mandatory in many domains such as healthcare (for medical treatment decisions), finance (for automated lending decisions), and autonomous vehicles (for decision-making by self-driving cars) where human lives depend on the decisions made by the ML model. Unfortunately, the decisions of complex models such as DNNs are difficult to explain.

Explainability is the ability to explain the decision-making process of an AI model in terms understandable to the end user~\cite{8631448}. An explainable model provides a clear and intuitive explanation of the decisions made, enabling users to understand why the model produced a particular result. In other words, explainability focuses on why an algorithm made a specific decision and how that decision can be justified.

Recent studies~\cite{Ribeiro2018,chan2022comparativestudyfaithfulnessmetrics,Singh1622975}
 in the domain of explainable ML outline two desired characteristics for explainers: human interpretable i.e., explanations must provide meaningful and qualitative understanding regarding the decision made by the model, by considering human’s limitations, and faithfulness i.e., explanations must correspond to how the model truly behaves. Guided by these desired characteristics, recent research has proposed various methods for explaining ML models. These methods range from feature attribution approaches such as SHAP~\cite{lundberg2017unifiedapproachinterpretingmodel} and LIME~\cite{Ribeiro2018}, example-based approaches such as counterfactual explanations~\cite{wachter2018CE}, and rule-based approaches such as forest-based trees~\cite{SAGI2020124} and DeepRED~\cite{wachter2018CE}.

Deep learning models, for instance, tend to be black boxes of the first kind because they are highly recursive. As the term is presently used in its most common form, an explanation is a separate model that is supposed to replicate most of the behaviour of a black box (for example, ‘the black box says that people who have been delinquent on current credit are more likely to default on a new loan’). Note that the term ‘explanation’ here refers to an understanding of how a model works, as opposed to an explanation of how the world works~\cite{Rudin2019}.

Automated vehicles are promising for decreasing traffic deaths and providing improved mobility, but also pose challenges in addressing the explainability of AI decisions. Autonomous vehicles have to make split-second decisions based on how they classify objects in the scene in front of them. The consequence can be dangerous if a self-driving car suddenly acts abnormally because of a misclassification problem. This is not merely a hypothetical scenario; such incidents are already occurring. For example, a self-driving Uber recently killed a woman in Arizona~\autocite{marshall2019uber}. It was the first known fatality involving a fully autonomous vehicle and a pedestrian. The information reported by anonymous sources claimed that the car software registered an object in front of the vehicle but treated it like a plastic bag or a tumbleweed carried in the wind. This incident underscores the urgent need for explainability in autonomous driving systems, as without a clear understanding of why the model made a particular decision, it is difficult to diagnose errors, improve system reliability, and ensure accountability. Explainable AI methods can provide critical insights into such failures, potentially preventing similar incidents in the future.


The lack of transparency in machine learning models raises significant challenges in domains where explainability is essential for trust and compliance. In safety-critical applications, blindly trusting model predictions is not an option—decisions must be understandable and justifiable. For example, in medical diagnosis, a deep learning model predicting a cancer diagnosis must provide clear reasoning behind its decision to assist doctors in making informed choices. In financial systems, AI models in loan approval processes must explain why an applicant was accepted or rejected to prevent biased decision-making. In security and law enforcement, AI models used for fraud detection or terrorism risk assessment must provide interpretable justifications to ensure fairness and reliability~\autocite{ribeiro2016ML}. An essential criterion for explainable models is that they should bridge the gap between input features and the model’s decision-making process. Ensuring explainability not only enhances trust and user acceptance but also helps in detecting biases, improving system robustness, and meeting regulatory requirements.

Efforts toward explainable AI for autonomous driving have gained momentum, with researchers exploring various interpretability techniques such as saliency maps, attention mechanisms, and rule-based systems~\autocite{bojarski2016endendlearningselfdriving, Ribeiro2018}. However, these approaches often lack actionability. They provide insights into why a decision was made but not how it could have been changed. Counterfactual explanations emerge as a viable approach by identifying the alterations required in input data to yield alternative results. Emerging predominantly in the late 2010s, these explanations aim to provide insights into machine learning decisions by exploring alternative scenarios. They present a human-friendly way to understand complex models \autocite{yeshwanth2023counterfactual}. For example, in the context of autonomous driving, a counterfactual explanation might illustrate how slight adjustments in sensor data, such as images captured by vehicle cameras, could lead to the vehicle stopping rather than proceeding forward. This technique not only offers valuable insights that can improve the model's efficacy and safety but also facilitates a deeper understanding of its decision-making mechanisms. Imagine an autonomous vehicle approaching a pedestrian or dog crossing. The vehicle decides to continue driving, but a counterfactual explanation reveals that if the image data from the camera showed a slightly different configuration of pixels (perhaps indicating the presence of a pedestrian or dog), the model would have decided to stop. This insight could prompt developers to adjust the model to be more sensitive to potential pedestrians, thereby enhancing safety.

This thesis explores counterfactual explanation generation using deep generative models, specifically Variational Autoencoders (VAEs), and compares different feature masking techniques to identify the most effective approach for improving explainability in autonomous driving.



% \section{Importance of Explainability in Machine Learning}

% Explainability is the ability to explain the decision-making process of an AI model in terms understandable to the end user \cite{8631448}. An explainable model provides a clear and intuitive explanation of the decisions made, enabling users to understand why the model produced a particular result. In other words, explainability focuses on why an algorithm made a specific decision and how that decision can be justified.

% Deep learning models, for instance, tend to be black boxes of the first kind because they are highly recursive. As the term is presently used in its most common form, an explanation is a separate model that is supposed to replicate most of the behaviour of a black box (for example, ‘the black box says that people who have been delinquent on current credit are more likely to default on a new loan’). Note that the term ‘explanation’ here refers to an understanding of how a model works, as opposed to an explanation of how the world works \cite{Rudin2019}.   


% Explainability in autonomous driving systems presents significant challenges due to the complexity and critical nature of real-time decision-making. Counterfactual explanations emerge as a viable approach by identifying the alterations required in input data to yield alternative results. Emerging predominantly in the late 2010s, these explanations aim to provide insights into machine learning decisions by exploring alternative scenarios. They present a human-friendly way to understand complex models \cite{yeshwanth2023counterfactual}. For example, in the context of autonomous driving, a counterfactual explanation might illustrate how slight adjustments in sensor data, such as images captured by vehicle cameras, could lead to the vehicle stopping rather than proceeding forward. This technique not only offers valuable insights that can improve the model's efficacy and safety but also facilitates a deeper understanding of its decision-making mechanisms. Imagine an autonomous vehicle approaching a pedestrian or dog crossing. The vehicle decides to continue driving, but a counterfactual explanation reveals that if the image data from the camera showed a slightly different configuration of pixels (perhaps indicating the presence of a pedestrian or dog), the model would have decided to stop. This insight could prompt developers to adjust the model to be more sensitive to potential pedestrians, thereby enhancing safety.


\section{Research Objectives}
The primary goal of this research is to develop a framework for generating counterfactual explanations using deep generative models, particularly Variational Autoencoders (VAEs), and to evaluate different feature masking strategies for improving interpretability in autonomous driving systems. This research aims to enhance the explainability of deep learning models by identifying how minimal changes in input features can alter classification outcomes, making AI-driven decisions more transparent and interpretable. Specifically, this research seeks to:

\begin{itemize}
    \item \textbf{Develop a VAE-based framework} for generating counterfactual explanations in deep learning models used in autonomous driving.
    The framework will leverage Variational Autoencoders (VAEs) to learn a latent space representation of input images, enabling the controlled modification of key features to generate counterfactual explanations. The proposed approach will integrate VAEs with a classifier to analyze changes in model predictions when specific latent features are altered. This process will allow for a systematic investigation of how the latent space can be used to generate human-interpretable counterfactuals in an autonomous driving context.

    \item \textbf{Investigate different feature masking techniques and evaluate their effectiveness in counterfactual generation.}  
    This research will explore multiple feature masking strategies to modify/mask input features and analyze their impact on counterfactual explanations. The methods investigated are detailed in the \cref{Methodology} section. The effectiveness of these feature masking techniques will be examined in terms of their ability to produce meaningful and realistic counterfactuals that lead to a change in model prediction.
    
    \item \textbf{Evaluate the counterfactual explanation framework using quantitative metrics.}  
    The generated counterfactuals will be assessed using well-established \textbf{quantitative evaluation metrics} to ensure that they are realistic, structurally similar to the original images, and capable of altering model predictions. These evaluations are clearly explained in \cref{Evaluation and Results}.

    \item \textbf{Conduct a human evaluation study to assess user preferences and interpretability of counterfactual explanations.}  
    Since counterfactual explanations are intended for human understanding, this research will conduct a user study to evaluate the effectiveness of different counterfactual generation techniques from a human perspective. Participants will be shown counterfactual explanations generated by different masking methods and will be asked to select the most preferred counterfactual explanation among different generated alternatives for the same image. They will then be asked to rate each counterfactual explanation based on the common counterfactual metrics like Interpretability, Plausibility, and visual Coherence. The results from this human evaluation will be compared with AI-based quantitative evaluation metrics to determine whether computational assessment aligns with human judgment.

\end{itemize}


\section{Research Questions}
To achieve the above objectives, this thesis seeks to answer the following key research questions:

\begin{itemize}
    \item \textbf{RQ1:} How can deep generative models effectively encode high-dimensional driving environment data to generate counterfactual explanations that improve autonomous driving interpretability?

    \item \textbf{RQ2:} How can loss function modifications in Variational Autoencoders (VAEs) optimize image reconstruction quality in autonomous driving tasks? 
    
    \item \textbf{RQ3:} How do different masking techniques impact the effectiveness and efficiency of counterfactual explanation generation, in terms of coverage, computational cost, method overlap, and failure rate?  (See \cref{Methodology} for details on the techniques investigated.) 

    \item \textbf{RQ4:} Which counterfactual explanation method is most preferred by users when selecting among generated explanations of the same original image? What factors (interpretability, plausibility, and visual coherence) influence user preference? (See \cref{Methodology} for details on the methods compared.)


\end{itemize}

\section{Thesis Structure (Outline of Chapters)}
This thesis is organized into multiple chapters, each addressing a key aspect of the research. \autoref{Introduction} introduces the motivation behind this work, outlines the problem statement, research objectives and questions, and provides an overview of the thesis structure. The
\autoref{Background} covers foundational concepts. It also introduces relevant evaluation metrics and properties and also applications.
The \autoref{Methodology} presents the core methodological contributions of the thesis. It details the dataset collection, the architecture and training process of the VAE and classifier, various feature masking techniques, and the process for generating counterfactual explanations. The \autoref{Evaluation and Results} reports both AI-based and human-centered evaluations. It presents different metrics and results, followed by an in depth user study and concludes with the overall discussion. The \autoref{Related work} chapter situates this thesis within the context of existing literature by reviewing and comparing prior work by other researchers. It highlights the key contributions of previous studies, identifies research gaps, and explains how this thesis builds upon and extends those works. Furthermore, it discusses how inspiration was drawn from existing methods and how this research addresses the identified limitations. Finally, \autoref{Conclusion and Future Work} summarizes the key contributions, discusses limitations, and outlines potential directions for future research.

\chapter{Background} \label{Background}
This section introduces notations and provides background on counterfactual explanations, Variational Autoencoders, LIME, Classifiers, and CARLA.

\section{Counterfactual Explanations}
In machine learning, decisions are often made by complex models without explicit explanations. For instance, imagine an autonomous vehicle approaching an intersection, relying on its vision system to classify traffic signs. If the vehicle recognizes a STOP sign, it will halt; however, if it misclassifies the same sign as a Speed Limit sign, this might lead to undesired consequences. Understanding precisely what minimal changes to the input would alter the model's decision is critical. This concept is precisely captured by \textit{counterfactual explanations}.

Formally, following Molnar~\cite{molnar2025}, a counterfactual explanation identifies the smallest possible modification to an input instance resulting in a different prediction outcome. Given a black-box classifier \( b \) that predicts an outcome \( y \) for an input instance \( x \):

\begin{equation}
y = b(x),
\end{equation}

a \textit{counterfactual explanation} finds an alternative instance \( x' \), such that:

\begin{equation}
b(x') \neq b(x),
\end{equation}

with the condition that the difference between \( x \) and \( x' \) is minimized according to a suitable distance metric.

For instance, in autonomous driving, if the classifier originally labels an image as:

\begin{equation}
b(x) = \text{"STOP Sign"},
\end{equation}

a valid and realistic counterfactual might suggest a slight reduction of brightness by 10\%, resulting in:

\begin{equation}
b(x') = \text{"Speed Limit Sign"}.
\end{equation}

\subsection{Desirable Properties of Counterfactual Explanations}
Counterfactual explanations are evaluated based on several key desirable properties, each ensuring usefulness, interpretability, and practicality:

\begin{enumerate}
    \item \textbf{Validity:}  
    Validity measures the proportion of generated counterfactuals that achieve the desired classification change:
    \begin{equation}
        b(x') \neq b(x).
    \end{equation}
    A higher validity score is preferable, indicating effective counterfactual generation.

    \item \textbf{Proximity (Minimality of Change):}  
    Counterfactual changes should be minimal, measured by a distance function \( d(x, x') \):
    \begin{equation}
        d(x, x') \text{ is minimized, with } d(x,x') < \delta,
    \end{equation}
    where \( \delta \) is a predefined threshold. Commonly used metrics include:
    \begin{itemize}
        \item \textbf{L1 norm (MAE)}: Sum of absolute differences.
        \item \textbf{L2 norm (MSE)}: Sum of squared differences.
        \item \textbf{Logcosh loss \cite{chen2019log}}: Smooth variant of MSE.
    \end{itemize}

    For instance, in autonomous driving, realistic counterfactual modifications involve adjusting brightness or contrast without changing essential features like the shape or position of a traffic sign. Lower values for proximity metrics indicate better counterfactual explanations.

    \item \textbf{Plausibility (Realism):}  
    Counterfactual instances must remain realistic and consistent with the original data distribution. For instance, suggesting a triangular STOP sign would be implausible since STOP signs are universally octagonal.

    \item \textbf{Actionability:}  
    Counterfactual recommendations must suggest feasible and realistic actions. For instance, suggesting increasing income or clearing existing debts is actionable for loan approval scenarios, whereas altering immutable features like age or gender is neither practical nor ethical.
\end{enumerate}

\subsection{Desirable Properties of Counterfactual Explainers}
Besides individual counterfactual properties, it is equally important to assess the algorithm generating them, referred to as a \textit{counterfactual explainer}. Desirable properties of a good counterfactual explainer include:

\begin{enumerate}
    \item \textbf{Efficiency:}  
    Counterfactual explanations must be generated rapidly, particularly for real-time decision-making scenarios like autonomous driving. An inefficient explainer can render explanations impractical for live diagnostic purposes.

    \item \textbf{Stability (Robustness):}  
    A stable explainer generates similar counterfactual explanations for similar input instances. If minor changes in input lead to significantly different explanations, trust in the explainer diminishes. Stability ensures reliability and consistency across explanations. High stability increases trust in the explainer and prevents inconsistencies in decision-making \cite{9660058} \cite{Kanamori:AISTATS2022}.

    \item \textbf{Fairness:}  
    Counterfactual explainers should not produce biased or unfair recommendations across different demographic groups. For instance, if a financial decision is influenced by demographic factors, leading to different counterfactual outcomes for similar input instances, the explainer fails the fairness criterion. Detailed discussions on fairness constraints in counterfactual explanations are provided by Kusner et al.~\cite{DBLP:journals/corr/abs-2010-10596} and many other authors.
\end{enumerate}

\subsection{Evaluation Metrics for Counterfactual Generation Algorithms}
Most counterfactual generation algorithms are evaluated based on the desirable properties mentioned above. Although the ideal evaluation would involve a user study to assess real-world usability, research typically uses quantifiable metrics instead. Common quantitative evaluation metrics include:

\begin{itemize}
    \item \textbf{Validity}: Percentage of generated counterfactuals successfully changing the class label.
    \item \textbf{Proximity:} Average distance between original and counterfactual instances, typically measured using L1, L2 norms, or similar metrics.
    \item \textbf{Plausibility:} Assessment of realism within the data distribution (e.g., reconstruction error using VAEs or distance to nearest neighbors).
    \item \textbf{Actionability:} Proportion of counterfactuals involving actionable changes.
    \item \textbf{Diversity:} Degree of variation across multiple counterfactual instances.
\end{itemize}

These metrics provide a systematic and practical way to assess the quality of generated counterfactual explanations.





















\section{Variational Autoencoders (VAE) for Representation Learning}
Generative modeling is an area of machine learning aimed at creating models capable of capturing and modeling complex probability distributions from observed datasets~\cite{doersch2016tutorial}. The overarching goal is to learn the underlying structure and dependencies of data, which is particularly useful in high-dimensional scenarios like images. For instance, in image generation, a generative model must understand intricate pixel-level relationships to produce realistic samples.

Variational Autoencoders (VAEs) represent a significant advancement in generative modeling by combining deep learning with principles from Bayesian inference. VAEs learn a probabilistic mapping from high-dimensional data to a structured latent space, facilitating both representation learning and data generation. This section provides an in-depth discussion of VAEs, covering their theoretical foundations, latent variable modeling, and applications.

\subsection{From Regular Autoencoders to Variational Autoencoders}
Traditional autoencoders (AEs) are designed to learn efficient low-dimensional representations of input data through a deterministic encoding-decoding process. The encoder compresses input data into a fixed-dimensional latent space, and the decoder reconstructs the original input from this compressed representation. While effective for feature extraction and data compression, traditional autoencoders lack a well-defined probabilistic structure, limiting their ability to generate diverse and novel samples. Variational Autoencoders (VAEs) address this limitation by introducing a probabilistic approach, where each input is mapped to a distribution in the latent space rather than a single point. This enables VAEs to generate new, meaningful samples by sampling from the learned latent distribution, making them powerful generative models.

\subsection{Foundations of Variational Autoencoders}
Variational Autoencoders, introduced by Diederik P. Kingma and Max Welling in their seminal 2013 paper~\cite{kingma2022autoencodingvariationalbayes} revolutionized the field of generative modeling. Unlike traditional autoencoders that primarily focus on data compression and reconstruction, VAEs remarkable ability to generate new data that resembles the training distribution. This capability stems from their unique approach to latent space representation, where inputs are mapped to probability distributions rather than fixed points. 

A VAE consists of two primary components: a probabilistic encoder and a probabilistic decoder. The encoder compresses input data into a latent representation, while the decoder reconstructs the original data from the latent variables. These two components are parameterized as neural networks and optimized using a variational inference approach.

\subsection{Variational Inference: An Approximate Inference Method}
Variational inference is a technique used in Bayesian statistics and machine learning to approximate intractable posterior distributions. In the context of VAEs, it is employed to approximate the true posterior distribution $p(z|x)$ of the latent variables given the observed data $x$, since direct computation is intractable. Instead of computing the exact posterior, VAEs introduce an approximate distribution $q_\phi(z|x)$, which is parameterized by the encoder network. The objective of variational inference is to find the optimal parameters $\phi$ such that $q_\phi(z|x)$ closely resembles the true posterior $p(z|x)$. This is achieved by minimizing the Kullback-Leibler (KL) divergence between the two distributions:
\begin{equation}
D_{KL}(q_\phi(z|x) || p(z|x)) = \mathbb{E}{q\phi(z|x)}[\log q_\phi(z|x) - \log p(z|x)].
\end{equation}
Since directly computing $p(x)$ is computationally expensive, variational inference instead maximizes the Evidence Lower Bound (ELBO), which is a surrogate objective that indirectly minimizes the KL divergence. By optimizing the ELBO, VAEs effectively learn to generate meaningful latent representations while ensuring that the approximate posterior remains close to the true posterior.

\subsection{Directed Probabilistic Models and Latent Variables}

VAEs are grounded in the framework of directed probabilistic models, commonly known as Bayesian networks. These models employ a directed acyclic graph (DAG) to represent dependencies among random variables. In VAEs, the generative process is modeled using a probabilistic graphical representation, where a latent variable $z$ is assumed to generate the observed data $x$.

Given an observed dataset $X = {x_1, x_2, ..., x_N}$, we assume that each data point $x_i$ is generated by a corresponding latent variable $z_i$. The generative process is characterized by a prior distribution over latent variables $p(z)$ and a likelihood model $p(x|z)$:
\begin{equation}
p(x) = \int p(x|z) p(z) dz.
\end{equation}
Since computing the exact posterior $p(z|x)$ is intractable due to the complexity of integrating over the latent space, VAEs leverage variational inference to approximate this posterior with a tractable distribution $q_\phi(z|x)$ parameterized by an encoder network.

\subsection{Variational Inference and the Evidence Lower Bound (ELBO)}
To approximate the posterior $p(z|x)$, VAEs introduce a variational distribution $q_\phi(z|x)$, typically chosen as a Gaussian distribution with mean $\mu_\phi(x)$ and variance $\sigma^2_\phi(x)$. The objective is to make $q_\phi(z|x)$ as close as possible to the true posterior by minimizing their Kullback-Leibler (KL) divergence~\cite{stackexchange_kl_vae}:
\begin{equation}
D_{KL}(q_\phi(z|x) || p(z|x)) = \mathbb{E}{q\phi(z|x)}[\log q_\phi(z|x) - \log p(z|x)].
\end{equation}
Since directly computing $p(x)$ is intractable, we maximize the Evidence Lower Bound (ELBO) instead:
\begin{equation}
\log p(x) \geq \mathbb{E}{q\phi(z|x)}[\log p_\theta(x|z)] - D_{KL}(q_\phi(z|x) || p(z)).
\end{equation}
The ELBO consists of two terms: a reconstruction term, which ensures that the decoded data resembles the input, and a regularization term, which encourages the approximate posterior $q_\phi(z|x)$ to be close to the prior $p(z)$. The prior is typically chosen as a standard normal distribution $\mathcal{N}(0, I)$ for simplicity and computational efficiency.


\subsection{Structure of Variational Autoencoders}
Figure \ref{fig:vae_structure} provides a schematic representation of the VAE architecture. A VAE comprises three main components:

\begin{figure}[htbp]
\centering
\includegraphics[width=0.7\textwidth]{img/vae/vae-gaussian.png}
\caption{Architecture of a Variational Autoencoder (VAE). The encoder maps inputs to a latent distribution; the decoder reconstructs data from samples drawn from this distribution~\cite{weng2018VAE}.}
\label{fig:vae_structure}
\end{figure}

\subsubsection{Encoder}
The encoder network maps input data  to a lower-dimensional latent space . Unlike traditional autoencoders, the encoder of a VAE outputs parameters defining a probabilistic distribution in latent space. Specifically, the encoder produces vectors representing the mean  and variance  for each latent dimension, defining a multivariate Gaussian distribution:
\begin{equation}
q(z|x) = \mathcal{N}(z; \mu(x), \Sigma(x)), \quad \text{where } \Sigma(x) = \text{diag}(\sigma_1^2, \dots, \sigma_n^2).
\end{equation}
\begin{figure}[htbp]
    \centering
    \includegraphics[width=0.7\textwidth]{img/vae/Image To Encoder To Latent Variable Parameters.png}
    \caption{Encoder network ($\phi$) mapping input $x$ to latent space parameters $(\mu_i, \sigma_i^2)$ for each latent dimension $z_i$. The encoder learns the approximate posterior distribution $q(z|x) = \mathcal{N}(z; \mu(x), \Sigma(x))$.}
    \label{fig:encoder}
\end{figure}

\subsubsection{Latent Variables}
Latent variables  are unobserved abstract representations inferred from the data during training. These variables provide a compressed, lower-dimensional abstraction of the data, capturing its essential features.

\subsubsection{Decoder}
The decoder reconstructs input data  from latent variables . It is designed probabilistically, modeling the conditional distribution . This setup allows the decoder to generate new data points resembling those from the training set by sampling from the latent distribution.
\begin{figure}[htbp]
    \centering
    \includegraphics[width=0.7\textwidth]{img/vae/decoder_figure.png}
    \caption{Decoder network ($\theta$) mapping latent variables $z$ to reconstructed output $x'$. The decoder models the likelihood $p(x'|z) = \mathcal{N}(x'; \mu_\theta(z), \sigma_\theta^2(z))$.}
    \label{fig:decoder}
\end{figure}


\subsection{VAE Objective Function and Loss}
The training objective of VAEs, derived from variational inference, maximizes the Evidence Lower Bound (ELBO):
\begin{equation}
\log p(x) \geq \mathbb{E}{q(z|x)}[\log p(x|z)] - D{KL}(q(z|x) \parallel p(z)).
\end{equation}

The first term is the reconstruction likelihood, encouraging accurate data reconstruction, while the second term ensures the approximate distribution  closely resembles the prior . Typically, a unit Gaussian  serves as the latent prior.

\subsection{Reparameterization Trick}
Training VAEs involves optimizing stochastic processes that require backpropagation, making direct sampling problematic. The "reparameterization trick" addresses this by decoupling randomness from network parameters. Instead of sampling directly from , we sample from a unit Gaussian and transform this random variable by the latent distribution's mean and variance:
\begin{equation}
z = \mu(x) + \sigma(x) \odot \epsilon, \quad \text{where } \epsilon \sim \mathcal{N}(0, I).
\end{equation}
This transformation ensures that gradients can flow through the latent variables, allowing efficient optimization using gradient-based learning.

As illustrated in Figure \ref{fig:reparameterization_trick}, the reparameterization trick decomposes the sampling process into deterministic and stochastic components. The encoder first predicts the mean  and standard deviation  of the latent distribution, and then a sample  is obtained by adding a scaled noise variable , where . This allows the latent space to remain continuous and differentiable while preserving the stochastic nature of the VAE.

Furthermore, Figure \ref{fig:reparameterization_backprop} demonstrates how the reparameterization trick enables backpropagation through stochastic nodes. Since the sampling operation is reformulated as a differentiable function of the noise variable , gradients can flow backward through  and , allowing efficient learning of the latent distribution parameters. This ensures that the encoder network effectively learns a structured latent space while maintaining a probabilistic formulation.

\begin{figure}[htbp]
    \centering
    \includegraphics[width=0.6\textwidth]{img/vae/reparameterization_trick.png}
    \caption{Visualization of the reparameterization trick: converting stochastic sampling into deterministic operations for effective backpropagation.Source Kingma and Weilling~\cite{Kingma_2019}}
    \label{fig:reparameterization_trick}
\end{figure}

\begin{figure}[htbp]
    \centering
    \includegraphics[width=0.7\textwidth]{img/vae/reparameterization_backprop.png}
    \caption{Illustration of gradient flow (backpropagation) enabled by the reparameterization trick in VAEs. Source Kingma and Weilling~\cite{Kingma_2019}}
    \label{fig:reparameterization_backprop}
\end{figure}


This transformation allows gradients to flow smoothly through the stochastic layers, enabling effective optimization.






\subsection{Applications of Variational Autoencoders}
VAEs have broad applicability across various domains:

\begin{itemize}
\item \textbf{Image Generation}: Creating new synthetic images with controlled variations.
\item \textbf{Data Denoising}: Removing noise from data by reconstructing from a clean latent representation.
\item \textbf{Dimensionality Reduction}: Compressing high-dimensional data into lower-dimensional representations for analysis and visualization.
\item \textbf{Anomaly Detection}: Identifying outliers by assessing reconstruction errors.
\end{itemize}

In summary, Variational Autoencoders bridge deep learning with probabilistic graphical models, allowing sophisticated representation learning and powerful generative capabilities, especially when handling complex, high-dimensional data like images.


















\section{Post-hoc Explainability Methods (LIME)}

Post-hoc methods are explainability techniques applied \textit{after} a model has made a prediction to understand the decision-making process. Unlike inherently interpretable models such as decision trees, many deep learning (DL) and machine learning (ML) models act as black boxes. Post-hoc explainability aims to bridge this gap by providing insights into the model’s predictions without modifying the model itself.

A common approach is \textbf{feature attribution}, which identifies the contribution of each input feature to a specific prediction. By doing so, we can determine the most influential features responsible for a decision.

\subsection{LIME: Local Interpretable Model-Agnostic Explanations}

LIME (Local Interpretable Model-Agnostic Explanations) is a widely used post-hoc explainability method introduced by Ribeiro et al. \cite{ribeiro2016ML}. It provides local explanations for individual predictions made by complex models. The key advantage of LIME is that it is model-agnostic, meaning it can be applied to any machine learning model. In the paper \cite{ribeiro2016ML} in which the authors propose a concrete implementation of local surrogate models. Surrogate models are trained to approximate the predictions of the underlying black box model. Instead of training a global surrogate model, LIME focuses on training local surrogate models to explain individual predictions.

LIME is motivated by the need to provide interpretable explanations for specific instances rather than understanding the entire model behavior globally. For instance, consider a healthcare dataset containing patient information such as BMI, weight, age, and blood group. Suppose a model predicts whether a patient is healthy or unhealthy. 

\begin{itemize}
    \item \textbf{Global explanation:} On average, BMI may be the most important feature influencing health, followed by weight.
    \item \textbf{Local explanation:} Suppose a 55-year-old patient is predicted as \textit{unhealthy}. LIME aims to explain which features (BMI, weight, blood group, etc.) contributed to this specific prediction and by what percentage.
\end{itemize}

In summary, LIME does not aim for a global understanding of the model but focuses on local explanations for individual predictions.

\textbf{Key Components of LIME:}
\begin{itemize}
    \item \textbf{Local:} Explanation is based on the neighborhood around the instance.
    \item \textbf{Interpretable:} A human should be able to understand the explanation.
    \item \textbf{Model-agnostic:} Can be applied to any ML model.
    \item \textbf{Explanations:} Identifies important features that influenced the prediction.
\end{itemize}

\subsubsection{Intuition Behind LIME}
Unlike deep learning (DL) models, simple linear models are inherently interpretable. 

Consider a basic linear regression model:
\begin{equation}
    y = w_1x_1 + w_2x_2 + w_3x_3 + \dots + w_nx_n
\end{equation}
where:
\begin{itemize}
    \item \( x_1, x_2, x_3, ..., x_n \) are input features,
    \item \( w_1, w_2, w_3, ..., w_n \) are the feature weights,
    \item \( y \) is the output prediction.
\end{itemize}

In this equation, the weights \( w_i \) directly tell us the importance of each feature \( x_i \). A high weight \( w_i \) indicates that feature \( x_i \) significantly influences the prediction.

However, complex models (deep neural networks, ensemble models, etc.) do not provide such direct interpretability. LIME approximates a black-box model locally using a simple linear model to extract interpretability.

\subsubsection{Mathematical Formulation of LIME}
LIME explains a prediction by fitting a local surrogate model around a given instance. Let:
\begin{itemize}
    \item \( f(x) \) be the black-box model making predictions.
    \item \( x' \) be the perturbed samples generated around the original instance \( x \).
    \item \( g(x') \) be the simple (interpretable) model approximating \( f(x) \) locally.
    \item \( \pi_x(x') \) be the proximity measure (distance function) that determines which samples are closer to \( x \).
\end{itemize}

LIME finds the best local explanation by solving the following optimization problem:
\begin{equation}
    \arg \min_{g \in G} L(f, g, \pi_x) + \Omega(g)
\end{equation}
where:
\begin{itemize}
    \item \( L(f, g, \pi_x) \) is the loss function that ensures \( g(x') \) closely approximates \( f(x') \) in the local neighborhood.
    \item \( \Omega(g) \) is a regularization term that keeps \( g \) simple (e.g., using fewer features).
    \item \( G \) is the set of all possible interpretable models (e.g., linear models, decision trees).
\end{itemize}

In simple terms:
- LIME generates perturbed versions of the input \( x \).
- It observes how the black-box model's predictions change.
- It fits a linear model \( g(x') \) to approximate the complex model locally.

\subsubsection{LIME on Tabular Data}
For tabular data, LIME works as follows:
1. Select an instance \( x \) for which we want an explanation.
2. Generate perturbed versions of \( x \) by randomly changing feature values.
3. Get predictions from the black-box model for these new perturbed samples.
4. Fit a simple linear model on the perturbed data and their predictions.
5. Extract feature weights to explain which features contributed most.

\textbf{Example: Loan Approval Prediction}
Consider a model that predicts whether a customer’s loan application will be approved. The dataset contains: Age, Income, Credit Score, and Loan Amount.


LIME can explain why a particular customer’s loan was \textit{denied}, showing the contribution of each feature.

\subsubsection{LIME on Images}
LIME for images works differently from tabular data. Instead of perturbing individual pixels, LIME:
\begin{description}
    \item 1. Segments the image into superpixels (groups of similar pixels)
    \item 2. Turns superpixels on/off by replacing them with a solid color (e.g., gray).
    \item 3. Runs the perturbed images through the black-box model.
    \item 4. Fits a linear model to approximate the model's behavior.
    \item 5. Identifies which superpixels contributed most to the classification.
\end{description}


\textbf{Example: Cat vs. Dog Classifier}

If the model predicts "Cat" for an image, LIME will identify which superpixels (fur, ears, eyes) are most responsible for the prediction. If a dog's ears are misclassified as a cat’s, LIME will highlight those regions.

% \subsubsection{LIME on Text Data}
% For text classification, LIME perturbs the input by randomly removing words and checking how the model’s prediction changes.
% \begin{enumerate}
%     \item Take a text sample (e.g., "This movie was absolutely amazing and fantastic!").
%     \item Remove different words to generate new versions of the text.
%     \item Observe how the model’s predictions change.
%     \item Assign importance scores to words based on their impact.
% \end{enumerate}
% For instance, in a sentiment analysis model, if removing "amazing" flips the prediction from positive to negative, then "amazing" is highly important.


LIME is a powerful tool for explaining black-box models. By approximating the decision boundary locally, it provides interpretable feature importance scores for individual predictions. It is widely used for debugging models, detecting bias, and improving trust in AI.

\subsubsection{Application domains of LIME}
One of the primary applications of LIME is in high-stakes decision-making environments where trust in AI systems is paramount. In healthcare, for instance, LIME can explain why a diagnostic model predicts a particular condition, helping physicians understand and verify the model's reasoning before making critical treatment decisions. Similarly, in financial services, LIME can explain loan approval or denial recommendations, ensuring decisions are based on legitimate factors rather than biased or irrelevant information. 








\chapter{Methodology} \label{Methodology}
This chapter describes the methodology used to generate counterfactual explanations using Variational Autoencoders (VAEs) and evaluate different feature masking strategies to enhance interpretability in autonomous driving systems. The methodology consists of multiple stages, including dataset collection, VAE training, classifier training, feature masking, counterfactual explanation generation.

A high-level workflow of the methodology is shown in Figure (let im edraw the methodology diagram and place an image). The dataset is collected from the CARLA simulator, preprocessed, and used to train a Variational Autoencoder (VAE) for latent space representation. A classifier is trained to distinguish between "STOP", "GO", "RIGHT, and "LEFT" decisions based on input images. Counterfactual explanations are generated by applying different feature masking techniques to alter the image or latent space representation. Finally, the counterfactuals are evaluated using both AI-based quantitative metrics and human evaluation studies.

\section{Experimental Setup}

All experiments in this thesis were conducted in a Linux environment to ensure compatibility with the CARLA simulator and associated tools. The setup was built using \textbf{CARLA version 0.9.15}, an open-source urban driving simulator widely adopted for autonomous driving research. This version provides a flexible and high-fidelity simulation environment, making it suitable for collecting diverse, labeled driving data under various rural and urban conditions.

To maintain compatibility with CARLA’s Python API, \textbf{Python version 3.7} was used for dataset collection. This version is recommended by CARLA’s developers to avoid API and dependency conflicts. The dataset was collected using two CARLA maps \textbf{Town03} and \textbf{Town07} \cite{CARLA2024}, provide varied driving topologies as shown in Figure~\ref{fig:carla_maps}. These maps were chosen due to their varied road layouts and environmental features, which provide rich scenarios for evaluating counterfactual explanations. Example simulation scenes demonstrating diverse driving and weather conditions are presented in Figure~\ref{fig:carla_scenes}. Users replicating this setup should download the CARLA server (v0.9.15) along with the additional maps from the \textit{official CARLA repository}. Once downloaded, the maps must be copied into the main CARLA directory to ensure proper integration.

\begin{figure}[htbp]
    \centering
    \begin{subfigure}{0.48\textwidth}
        \centering
        \includegraphics[width=\linewidth]{img/carla/town03.png}
        \caption{CARLA Town03 layout}
    \end{subfigure}
    \hfill
    \begin{subfigure}{0.48\textwidth}
        \centering
        \includegraphics[width=\linewidth]{img/carla/town_7.png}
        \caption{CARLA Town07 layout}
    \end{subfigure}
    \caption{Top-down map layouts of the selected CARLA towns used for dataset collection~\cite{CARLA2024}}.
    \label{fig:carla_maps}
\end{figure}

\begin{figure}
    \centering
    \includegraphics[width=0.7\linewidth]{img/carla/CARLA_Environment.png}
    \caption{3D Sample scenes from CARLA showcasing diverse conditions during dataset collection. The images illustrate a variety of urban and rural settings, with diverse lighting and weather conditions, including sunny, foggy, rainy, and snowy scenarios. Each scene highlights the flexibility of CARLA in simulating realistic driving environments for autonomous vehicle testing.}
    \label{fig:carla_scenes}
\end{figure}


Before executing any client-side scripts for dataset collection, it is essential to start the CARLA server using the following command in the CARLA root directory: ./CarlaUE4.sh


This launches the CARLA simulation in the Unreal Engine environment. For smooth operation, it is recommended to run the project on a system with at least \textbf{16 GB of RAM}, a dedicated GPU (e.g., \textbf{NVIDIA RTX series}), and \textbf{Ubuntu 18.04 or 20.04 LTS}. 

For the development and training of the Variational Autoencoder (VAE) and classifier models, we used \textbf{Python version 3.11 or higher}. This was necessary to avoid compatibility issues with newer versions of PyTorch, which are not well supported on older Python versions like 3.7. While dataset collection was handled using Python 3.7, the core machine learning model development required a more modern Python environment.  

To monitor and interpret various aspects of model training such as loss, accuracy, weight distributions, and layer activations we used \textbf{TensorBoard}. It provided valuable insights into training behavior and helped fine-tune model performance.

\section{Dataset Collection, Labeling, and Splitting Process}

To train and evaluate the proposed models effectively, a high-quality and well-structured dataset was essential. A systematic data preparation pipeline consisting of three key stages \textit{collection}, \textit{labeling}, and \textit{splitting}. The dataset was first collected using the CARLA simulator, which offers a controllable and realistic environment for simulating diverse driving scenarios. Following data collection, a precise rule-based labeling strategy was applied using vehicle control signals to assign meaningful class labels. Finally, the labeled dataset was partitioned into training and testing subsets to ensure class balance and prevent data leakage. This structured pipeline ensured consistency, and reproducibility  throughout the experimental workflow.

\subsection{Dataset Collection}

The dataset was collected using the CARLA simulator~\cite{CARLA2024docs}. An Audi A2 vehicle was deployed in autopilot mode to autonomously navigate the environment while capturing RGB images and corresponding control signals. A front mounted RGB camera was configured with a 125° field of view (FoV) and a resolution of $160 \times 80$ pixels. This resolution was selected to balance computational efficiency with sufficient visual detail for model learning.

A total of approximately 12,000 images were collected under varied driving scenarios. Each image was paired with the vehicle’s control parameters steering, throttle, and brake. The \textit{steering angle} ranged from $-1$ (full left) to $1$ (full right), while \textit{throttle} and \textit{brake} values ranged from $0$ to $1$, representing the intensity of acceleration and braking, respectively. This multi-modal data captured both visual context and driving behavior. The resulting dataset served as the foundation for both binary (STOP vs. GO) and multi-class (STOP, GO, LEFT, RIGHT) classification tasks.


\subsection{Dataset Labeling}

The collected data was labeled using a deterministic rule-based strategy derived from the vehicle’s control inputs. Two distinct labeling schemes were implemented to support binary and multi-class classification objectives.

For the \textbf{binary-class scheme}, each frame was labeled as either \texttt{STOP} or \texttt{GO}. A frame was labeled as \texttt{STOP} if the brake value exceeded a predefined threshold. Otherwise, it was labeled as \texttt{GO}. This distinction effectively captured the vehicle's motion state based on braking behavior.

For the \textbf{multi-class scheme}, labels were assigned based on prioritized control logic:
\begin{itemize}
    \item STOP, if the brake value exceeded a defined threshold.
    \item RIGHT, if the steering value was significantly positive and throttle was active.
    \item LEFT, if the steering value was significantly negative and throttle was active.
    \item GO, for all remaining cases where the vehicle moved straight without braking or significant steering.
\end{itemize}

This approach ensured mutually exclusive and semantically meaningful labels for each image. Threshold values for labeling were empirically determined based on the distribution of control signals across the dataset. The process was fully automated, ensuring reproducibility and eliminating manual labeling bias.

\subsection{Dataset Splitting}

Following labeling, the dataset was divided into separate training and testing subsets. The splitting was performed \textit{after} labeling to maintain label integrity and avoid any form of data leakage. Care was taken to ensure that the distribution of class labels remained balanced across both sets. This was critical for promoting fair learning and evaluation, particularly in the multi-class scenario.

The labeled dataset was partitioned using an 80/20 split ratio, where 80\% of the data was used for training and 20\% for testing. This ensured sufficient data for model learning while preserving a representative set for evaluation.

For the binary classification scheme, the training set contained 4,959 GO and 4,728 STOP samples, while the test set included 1,262 GO and 1,160 STOP samples, maintaining a near-balanced distribution across classes. Similarly, for the 4-class setting, all classes (GO, STOP, LEFT, RIGHT) were equally represented with 3,327 samples in training and 821 samples per class in testing. The distributions for both schemes are visualized side by side in Figure~\ref{fig:label_distribution_combined}.


\begin{figure}[htbp]
    \centering
    \begin{subfigure}{0.48\textwidth}
        \centering
        \includegraphics[width=\linewidth]{img/dataset/2class_distribution.png}
        \caption{Binary (GO/STOP) label distribution\vphantom{g}}
        \label{fig:binary_class_distribution}
    \end{subfigure}
    \hfill
    \begin{subfigure}{0.5\textwidth}
        \centering
        \includegraphics[width=\linewidth]{img/dataset/4class_distribution.png}
        \caption{4-class label distribution (GO, STOP, LEFT, RIGHT)}
        \label{fig:class_distribution_pie}
    \end{subfigure}
    \caption{Label distributions across training and testing datasets for binary and multi-class classification schemes.}
    \label{fig:label_distribution_combined}
\end{figure}





\section{Variational Autoencoder (VAE)} \label{sec:vae}

This section outlines the architectural design, training process, and evaluation protocol for the Variational Autoencoder (VAE) used in this work. The VAE serves as a generative backbone for producing realistic and plausible counterfactual explanations in the context of autonomous driving. Each component of the system is described in detail, along with the rationale for the chosen configurations and metrics.

To generate semantically coherent and visually realistic counterfactual explanations that remain on the data manifold, we adopt a Variational Autoencoder (VAE) as the generative backbone of our system. This approach draws conceptual motivation from the Contrastive Explanation Method (CEM)~\cite{DBLP:journals/corr/abs-1802-07623}, which leverages autoencoders to constrain generated samples within the support of the data distribution. While CEM was originally evaluated on low-dimensional datasets such as MNIST, our application domain involves high-resolution RGB images from complex driving scenarios. Consequently, the VAE architecture, training objective, and optimization strategy were carefully adapted and refined over multiple design iterations.

\subsection{VAE Architecture} \label{sec:vae_architecture}

The VAE consists of two primary components, a convolutional encoder that transforms the input image into a latent distribution, and a decoder that reconstructs the image from a sampled latent vector. The encoder and decoder are trained jointly using a variational loss function that enforces both reconstruction fidelity and latent space regularization.

\subsubsection{Encoder Architecture} \label{subsubsec:vae_encoder}

The encoder takes an input image of shape $3 \times 80 \times 160$ (RGB, height $\times$ width) and compresses it into a 128-dimensional latent space. The architecture is composed of four convolutional blocks followed by fully connected layers:

\begin{itemize}
    \item \textbf{Conv Layer 1:} 64 filters, kernel size $4 \times 4$, stride 2, no padding. Followed by LeakyReLU.
    \item \textbf{Conv Layer 2:} 128 filters, kernel size $3 \times 3$, stride 2, padding 1. Followed by BatchNorm and LeakyReLU.
    \item \textbf{Conv Layer 3:} 256 filters, kernel size $4 \times 4$, stride 2. Followed by LeakyReLU.
    \item \textbf{Conv Layer 4:} 512 filters, kernel size $3 \times 3$, stride 2. Followed by BatchNorm and LeakyReLU.
\end{itemize}

The final feature map output is flattened and passed through a fully connected layer with 1024 units (with LeakyReLU activation). From this, two separate linear layers output the mean vector $\mu \in \mathbb{R}^{128}$ and log-variance vector $\log\sigma^2 \in \mathbb{R}^{128}$, representing the parameters of the approximate posterior $q(z|x)$.

\subsubsection{Latent Sampling via Reparameterization Trick}
The latent vector $z$ is sampled using the reparameterization trick:
\begin{equation}
z = \mu + \sigma \cdot \epsilon, \quad \epsilon \sim \mathcal{N}(0, I)
\end{equation}
This allows gradient-based optimization through the stochastic sampling process, maintaining end-to-end differentiability. The encoder uses PyTorch’s native implementation of a normal distribution to sample $\epsilon$, and places the distribution tensors on the correct device (CPU or GPU).




\subsubsection{Decoder Architecture} \label{subsubsec:vae_decoder}

The decoder reverses the encoding process and reconstructs an image from the latent vector $z \in \mathbb{R}^{128}$. It consists of two fully connected layers followed by four transposed convolutional layers:

\begin{itemize}
    \item \textbf{Dense Layer 1:} Linear projection from 128 to 1024 units with LeakyReLU.
    \item \textbf{Dense Layer 2:} Linear projection from 1024 to $512 \times 4 \times 9 = 18432$ units, reshaped to $(512, 4, 9)$.
\end{itemize}

The reshaped feature map is then passed through:

\begin{itemize}
    \item \textbf{Deconv Layer 1:} 256 filters, kernel size $4 \times 4$, stride 2, padding 1, output padding (0,1). Followed by LeakyReLU.
    \item \textbf{Deconv Layer 2:} 128 filters, kernel size $4 \times 4$, stride 2, padding 1, output padding (1,1). Followed by LeakyReLU.
    \item \textbf{Deconv Layer 3:} 64 filters, kernel size $4 \times 4$, stride 2. Followed by LeakyReLU.
    \item \textbf{Deconv Layer 4:} 3 filters (RGB), kernel size $4 \times 4$, stride 2. Followed by Sigmoid activation.
\end{itemize}

The final output has shape $3 \times 80 \times 160$, matching the input dimensions. The Sigmoid activation ensures output values lie in the normalized $[0,1]$ range.

\subsubsection{Training Objective and Optimization Strategy} \label{subsubsec:vae_loss}

The training objective for the VAE is the variational loss function:
\[
\mathcal{L}_{\text{VAE}} = \mathcal{L}_{\text{recon}} + \lambda_{\text{KL}} \cdot \mathcal{L}_{\text{KL}}
\]

\paragraph{Reconstruction Loss:} \label{reconstruction_loss}
Two reconstruction losses were implemented and tested:
\begin{itemize}
    \item \textbf{Mean Squared Error (MSE):} Penalizes squared differences between the original and reconstructed pixel values. Suitable for pixel-level fidelity.
    \item \textbf{Log-Cosh Loss:} More robust to outliers, behaves like MSE for small errors and like MAE for large errors.
\end{itemize}
The use of Log-Cosh was empirically found to produce smoother reconstructions in early training phases. This is motivated by Chen et al.~\cite{chen2019log}. So, in this thesis comparison between the MSE loss and Log-cosh loss was implemented and compared between them. The results are discussed in the     .

\paragraph{KL Divergence:}
The KL divergence regularizes the latent distribution:
\[
\mathcal{L}_{\text{KL}} = -\frac{1}{2} \sum_{i=1}^{d} \left(1 + \log\sigma_i^2 - \mu_i^2 - \sigma_i^2\right)
\]
It encourages the approximate posterior $q(z|x)$ to stay close to the unit Gaussian prior $p(z) = \mathcal{N}(0, I)$.

\paragraph{KL Weight Annealing:}
To avoid early dominance of the KL term, we implement linear annealing:
\[
\lambda_{\text{KL}} = \min(\lambda_0 + \delta \cdot \text{epoch}, 1.0)
\]
where $\lambda_0 = 5 \times 10^{-5}$ and $\delta = 1 \times 10^{-4}$. This allows the model to prioritize reconstruction initially, and gradually introduce regularization.

\paragraph{Training Configuration:}
The VAE is implemented using PyTorch and trained in Python 3.11 on a Linux system an NVIDIA CUDA-enabled GPU (e.g., RTX 1080 Ti). Libraries include Torchvision, Matplotlib, NumPy, and PIL.

\begin{itemize}
    \item Optimizer: Adam
    \item Learning Rate: $1 \times 10^{-4}$; Weight Decay: $1 \times 10^{-5}$
    \item Batch Size: 128
    \item Epochs: 200
    \item Scheduler: \texttt{ReduceLROnPlateau} (patience = 10, factor = 0.5)
    \item Early Stopping: patience = 50
\end{itemize}

Each epoch logs training and validation losses (total, reconstruction, KL), pixel-level accuracy, and PSNR.

\subsection{Evaluating VAE Performance} \label{subsec:vae_evaluation}

Model performance is assessed using both quantitative and qualitative metrics:
\begin{itemize}
\item \textbf{Reconstruction Loss:} MSE or Log-Cosh.
\item \textbf{Pixel Accuracy:} Proportion of correctly reconstructed pixels after thresholding.
\item \textbf{PSNR:} Perceptual quality of reconstruction.
\item \textbf{Visual Inspection:} Periodic reconstruction samples are compared with ground truth.
\item \textbf{Latent Space Evaluation:} Interpolation and prior sampling validate structure and continuity.
\item \textbf{Training Curves:} Epoch-wise plots for all metrics enable trend analysis and convergence monitoring.
\end{itemize}

These evaluations confirm the VAE's capacity to learn expressive, structured representations, which are later used for counterfactual explanation in \autoref{Evaluation and Results}.



\section{Classifier Model for Prediction} \label{sec:classifier_mdel_for_prediction}
To assess the semantic structure of the learned latent space and evaluate downstream task performance, multiple classifiers were trained using the 128-dimensional latent representations generated by the VAE encoder. Each image corresponds to a driving situation (e.g., STOP, GO, RIGHT, LEFT), and the classifiers were trained to predict the appropriate driving action class from the latent vector.

Unlike conventional approaches where classifiers are trained directly on pixel-level image data, this work emphasizes feature-based classification, operating solely in the compressed and semantically meaningful latent space. This allows for efficient model training, improved interpretability, and compatibility with counterfactual explanation techniques explored later in the thesis.

Five classification models were implemented and evaluated:
\begin{itemize}
    \item \textbf{Neural Network (MLP):} A 3-layer fully connected network with LeakyReLU activations and dropout.
    \item \textbf{Logistic Regression:} A linear baseline model to establish separability in the latent space.
    \item \textbf{K-Nearest Neighbors (KNN):} A non-parametric, distance based model suitable for evaluating the local structure of the latent space.
    \item \textbf{Support Vector Machine (SVM):} A margin based classifier using a radial basis function kernel for better non-linear decision boundaries.
    \item \textbf{Random Forest:} An ensemble based decision tree model offering robustness and interpretability.
\end{itemize}

All classifiers were trained in both binary and multi-class settings. The binary setup involved classifying between \texttt{GO} and \texttt{STOP} scenarios, while the multi-class setup extended this to include \texttt{LEFT} and \texttt{RIGHT} actions, resulting in a four-class prediction task.

As detailed in Section~\ref{subsubsec:vae_encoder}, the encoder compresses each $3 \times 80 \times 160$ image into a 128-dimensional latent vector using a series of convolutional and fully connected layers. These latent vectors serve as inputs to all classifiers, forming the basis for prediction.

No pixel-level image information was used during classification. The models operate solely on the semantically rich latent space learned by the VAE. This aligns with the overall objective of this work: to build interpretable and compact representations that support both classification and counterfactual explanation generation.

\subsection{Classifier Architectures and Training Setup}

This section describes the architecture, training configurations, and optimization strategies used for each classifier. All models were trained and evaluated using the latent feature vectors extracted from the VAE encoder. The input to each classifier is a 128-dimensional feature vectors.

\subsubsection*{Neural Network Classifier (MLP)}

A deep feedforward neural network was implemented using PyTorch to classify the latent vectors extracted from the VAE. The architecture was designed with simplicity and effectiveness in mind, specifically tailored to the 128-dimensional latent representations. This classifier is composed of three fully connected (dense) layers, each followed by batch normalization and dropout regularization, with LeakyReLU activations throughout. The details are as follows:

\begin{itemize}
    \item \textbf{Input size:} 128 \\
    The input to the network is a 128-dimensional vector, corresponding to the latent features produced by the VAE encoder. This compact representation captures the essential semantic information from the original image.
    
    \item \textbf{Hidden layers:} 3 layers, each with 128 units \\
    Three fully connected hidden layers were chosen to provide the network with sufficient capacity to model non-linear relationships within the latent space. Maintaining a consistent layer size (128 units) aligns with the dimensionality of the latent vector, ensuring that each hidden layer can process the full feature set without dimensionality reduction, which helps preserve the rich information embedded in the latent space.
    
    \item \textbf{Activation:} LeakyReLU with slope 0.01 \\
    LeakyReLU is used as the activation function for its advantages over the traditional ReLU. Unlike ReLU, which zeroes out negative inputs, LeakyReLU allows a small, non-zero gradient (with a slope of 0.01) for negative values. This feature mitigates the "dying ReLU" problem, ensuring that neurons do not become inactive during training and thereby promoting better gradient flow across the network.
    
    \item \textbf{Regularization:} Dropout (p = 0.5) and Batch Normalization \\
    Dropout is applied with a probability of 0.5 after each hidden layer to prevent overfitting by randomly deactivating half of the neurons during training. This forces the network to learn more robust features that are not overly dependent on any single neuron. Batch Normalization is applied to stabilize and accelerate training by normalizing the inputs to each layer. It reduces internal covariate shift, which helps in achieving faster convergence and improved overall performance.
    
    \item \textbf{Output:} 2 or 4 logits depending on classification task \\
    The final output layer produces either 2 or 4 logits, corresponding to the binary or multi-class classification tasks, respectively. The logits represent the unnormalized scores for each class, and a softmax function is applied during loss computation (using CrossEntropy Loss) to obtain probability distributions over the classes.
\end{itemize}

This architecture was selected because it effectively balances model complexity and computational efficiency, making it well-suited for the compact and informative VAE latent space. By leveraging non-linear activations and regularization techniques, the network is capable of learning subtle distinctions between classes while mitigating overfitting, ultimately yielding strong classification performance.



\subsubsection*{Traditional Classifiers}

To compare the neural classifier's performance with classical machine learning models, the following classifiers were implemented using the scikit-learn library:

\begin{itemize}
    \item \textbf{Logistic Regression:} A linear classifier using the liblinear
    
    
    solver and L2 regularization.
    \item \textbf{K-Nearest Neighbors (KNN):} Set to $k=5$ using Euclidean distance.
    \item \textbf{Support Vector Machine (SVM):} RBF kernel with probability estimates enabled.
    \item \textbf{Random Forest:} Ensemble of 100 decision trees with a maximum depth determined automatically.
\end{itemize}

These models were trained on the same latent vectors as the neural network. The dataset was split into 80\% training and 20\% testing for evaluation. Metrics such as precision, recall, F1-score, confusion matrix, and ROC-AUC were computed for comprehensive analysis.

\subsubsection*{Binary vs Multi-Class Setup}

All classifiers were trained in two different settings:
\begin{itemize}
    \item \textbf{Binary Classification:} Predicting whether to \texttt{GO} or \texttt{STOP}, useful for early validation and simple decision-making.
    \item \textbf{Multi-Class Classification:} Predicting one of four classes: \texttt{STOP}, \texttt{GO}, \texttt{RIGHT}, or \texttt{LEFT}, enabling fine-grained driving behavior modeling.
\end{itemize}

Each setup used the same pipeline of latent feature extraction followed by classification. Evaluation was performed using the same data split and metrics across all models to enable a fair comparison.


\section{Feature Masking Techniques for Counterfactual Generation}
\label{sec:feature_masking_pipeline}


To generate counterfactual explanations (CEs), multiple feature masking strategies were employed, each with its own rationale, masking space (image or latent), and mechanism of perturbation. The objective in all these methods is consistent: identify minimal and plausible changes in the input that result in a different classification output, thus providing insight into the model decision boundary.

All methods follow a unified processing pipeline comprising the following steps:
\begin{enumerate}
    \item Encode the original image using a Variational Autoencoder (VAE) to obtain a latent representation.
    \item Classify the latent vector using a trained classifier to obtain the original prediction.
    \item Apply a masking strategy (in image space or latent space).
    \item Reconstruct the masked or modified input using the decoder.
    \item Re-encode and classify the reconstructed image.
    \item Compare the new prediction with the original; if different, a counterfactual explanation is identified.
\end{enumerate}

The masking strategies are categorized into two groups based on the space where the perturbation is applied:

\subsection*{Image Space Masking}
This includes Grid-Based Masking, LIME on Images, and Object Detection-Based Masking. These methods apply masking directly to the input image:
\begin{itemize}
    \item \textbf{Grid-Based Masking:} The image is divided into grids (e.g., 10$\times$5, 4$\times$2), and each cell is masked iteratively to observe prediction changes.
    \item \textbf{LIME on Images:} LIME is applied on pixel space to identify important regions, which are then masked.
    \item \textbf{Object Detection-Based Masking:} YOLOv5 is used to detect semantic objects (e.g., pedestrians, vehicles), which are then removed from the image by zeroing out pixels in the bounding box.
\end{itemize}

\subsection*{Latent Space Masking}
This includes LIME on Latent Features and LIME with Nearest Unchanged Neighbor (NUN). Here, perturbations are applied to the encoded latent vector:
\begin{itemize}
    \item \textbf{LIME on Latent Features:} LIME identifies influential latent dimensions. These are replaced using strategies such as median substitution or rule-based adjustments using dataset statistics.
    \item \textbf{LIME with NUN:} Combines LIME-based feature importance with the Nearest Unchanged Neighbor strategy, selecting feature replacements that are more robust and semantically meaningful.
\end{itemize}

For each method, similarity metrics (SSIM, PSNR, MSE, UQI, VIFP) are computed between the original and reconstructed images to assess visual fidelity. All results, including prediction changes, classifier confidences, masking parameters, and processing time, are logged into method-specific files for analysis. If the prediction changes after masking, the example is labeled as a successful counterfactual. This analysis directly supports answering the research question outlined in \cref{sec:research_question}, particularly \textbf{RQ3}.


In the subsequent subsections, each masking method is detailed along with its algorithm, implementation logic.
 

\subsection{Grid-Based Masking} \label{sec:grid_based_masking}

Grid-based masking is a spatial perturbation technique that aims to generate counterfactual explanations (CEs) by selectively removing small rectangular regions from the input image. The assumption is that certain spatial locations contribute more significantly to the model’s decision, and masking those regions may cause the classifier to change its prediction.

As with all image space masking strategies, the method follows the general processing pipeline described in Section~\ref{sec:feature_masking_pipeline}. The original image is first encoded into a latent representation using a Variational Autoencoder (VAE), classified to obtain the original prediction, and decoded to reconstruct the image. The masking operation is then applied directly on the input image.

This method uses two grid configurations:
\begin{itemize}
    \item \textbf{Fine grid:} $10 \times 5$ (smaller regions)
    \item \textbf{Coarse grid:} $4 \times 2$ (larger regions)
\end{itemize}

The masking starts with the finer grid to enable high-resolution localization. If a counterfactual is found during this stage, i.e., if masking any one grid cell leads to a prediction change, the process stops early (early termination). If no CE is found, the method proceeds to the coarser grid as a fallback. Each grid cell is masked by setting its pixel values to zero (blackout), and then the image is passed through the reconstruction and classification steps to evaluate its impact.

If the classifier’s prediction changes after reconstruction, a counterfactual explanation is considered successfully generated. In such cases, image quality metrics (e.g., SSIM, PSNR, MSE, UQI, VIFP) are computed to assess how much the reconstruction deviates from the original. If a counterfactual is detected (i.e., a change in prediction), image quality metrics (SSIM, PSNR, etc.) are computed, and the original, masked, and reconstructed images are saved. All relevant metadata is logged for evaluation.


\vspace{1em}
\begin{algorithm}[H]
\caption{Grid-Based Masking for Counterfactual Generation}
\label{alg:grid_based_masking}
\begin{algorithmic}[1]
\REQUIRE Image $I$, encoder $E$, decoder $D$, classifier $C$, grid configurations $G = \{(10,5), (4,2)\}$
\ENSURE Whether a counterfactual explanation is found

\STATE Encode image: $z \leftarrow E(I)$
\STATE Predict original label: $y_{\text{orig}} \leftarrow \arg\max C(z)$

\FOR{each grid size $(m, n)$ in $G$}
    \STATE $T \leftarrow m \times n$
    \FOR{$p = 0$ to $T - 1$}
        \STATE Mask grid cell $p$ in $I$ $\rightarrow I_{\text{masked}}$
        \STATE $z_{\text{masked}} \leftarrow E(I_{\text{masked}})$
        \STATE $I_{\text{recon}} \leftarrow D(z_{\text{masked}})$
        \STATE $z_{\text{re}} \leftarrow E(I_{\text{recon}})$
        \STATE $y_{\text{new}} \leftarrow \arg\max C(z_{\text{re}})$
        \IF{$y_{\text{new}} \neq y_{\text{orig}}$}
            \RETURN Counterfactual explanation found
        \ENDIF
    \ENDFOR
\ENDFOR

\RETURN No counterfactual explanation found
\end{algorithmic}
\end{algorithm}

\vspace{1em}

This method provides a simple yet effective baseline for comparison with more advanced techniques. Its interpretability and ability to localize spatial influence make it particularly suitable for safety-critical domains such as autonomous driving.



\subsection{Object Detection-Based Masking} \label{sec:object_detection_masking}

Object Detection-Based Masking leverages semantic object detection to generate counterfactual explanations by selectively removing real-world objects from the input image. The motivation is to evaluate how the presence or absence of specific objects—such as pedestrians, vehicles, or road signs—affects the model’s decision.

As with all image-space methods, the process follows the unified pipeline described in Section~\ref{sec:feature_masking_pipeline}. The input image is first encoded into a latent vector using a Variational Autoencoder (VAE), then classified to obtain the original label and decoded to reconstruct the image. The masking operation, however, is guided by an object detector rather than a fixed spatial grid.

In this implementation, a pre-trained YOLOv5 model is used to detect objects in the image. Each detected object is represented by a bounding box. The pixels within the bounding box are zeroed out, effectively removing the object from the image. Only the first detected object is considered per image to maintain efficiency and interpretability.

The masked image is passed through the encoder-decoder-classifier pipeline. If the classifier’s prediction changes compared to the original, a counterfactual explanation (CE) is identified. As with other methods, the image quality metrics (SSIM, PSNR, MSE, UQI, VIFP) are computed, and the original, masked, and reconstructed images are saved. All metadata, including the detected object label and bounding box coordinates, are logged.

\vspace{1em}
\begin{algorithm}[H]
\caption{Object Detection-Based Masking for Counterfactual Generation}
\label{alg:object_detection_masking}
\begin{algorithmic}[1]
\REQUIRE Image $I$, encoder $E$, decoder $D$, classifier $C$, object detector $Y$
\ENSURE Whether a counterfactual explanation is found

\STATE Encode image: $z \leftarrow E(I)$
\STATE Predict original label: $y_{\text{orig}} \leftarrow \arg\max C(z)$
\STATE Detect objects using YOLO: $\mathcal{B} \leftarrow Y(I)$

\IF{$\mathcal{B}$ is empty}
    \RETURN No objects detected; no masking applied
\ENDIF

\FOR{bounding box $(x_{\min}, y_{\min}, x_{\max}, y_{\max}) \in \mathcal{B}$}
    \STATE Mask region in $I$ by setting pixels to 0 within the bounding box → $I_{\text{masked}}$
    \STATE $z_{\text{masked}} \leftarrow E(I_{\text{masked}})$
    \STATE $I_{\text{recon}} \leftarrow D(z_{\text{masked}})$
    \STATE $z_{\text{re}} \leftarrow E(I_{\text{recon}})$
    \STATE Predict new label: $y_{\text{new}} \leftarrow \arg\max C(z_{\text{re}})$
    \IF{$y_{\text{new}} \neq y_{\text{orig}}$}
        \RETURN Counterfactual explanation found
    \ENDIF
    \STATE \textbf{break} \COMMENT{Only mask first detected object}
\ENDFOR

\RETURN No counterfactual explanation found
\end{algorithmic}
\end{algorithm}
\vspace{1em}

This method introduces semantic understanding into counterfactual generation by tying the prediction to identifiable objects in the scene. It is particularly relevant in safety-critical applications such as autonomous driving, where the influence of pedestrians, vehicles, or signs on model behavior must be made transparent.



\subsection{LIME on Images} \label{sec:lime_on_images}
Local Interpretable Model-Agnostic Explanations (LIME) is an explainability technique that identifies influential regions in an input image based on perturbation and model response. In this method, LIME is applied directly to the input image to generate counterfactual explanations by masking important image segments and observing if the prediction changes.

Following the unified pipeline described in Section~\ref{sec:feature_masking_pipeline}, the original image is first encoded using a Variational Autoencoder (VAE), classified to obtain the original label, and decoded for reconstruction. LIME is then applied to the raw image to identify the most salient superpixels (image patches) contributing positively to the model's decision.

To ensure visual consistency and interpretability, a soft masking approach is employed using \textit{alpha blending}, where the selected regions are blended with a black background instead of being fully removed. This avoids abrupt pixel transitions and ensures that the masked input remains within the training distribution.

The LIME-identified masked image is then encoded, decoded, and re-classified. If the classification changes compared to the original, a counterfactual explanation is considered found. As in other methods, similarity metrics such as SSIM, PSNR, and MSE are computed to evaluate the visual fidelity of the reconstructed image, and all relevant data is logged for evaluation.


Although LIME is applied directly to the input image, the model it explains is not the VAE itself but the downstream classifier. The predictive function passed to LIME consists of two components: first, the input image is encoded into a latent representation using the VAE encoder; second, the latent vector is passed to the trained classifier to produce a prediction. LIME then perturbs the image (e.g., by masking superpixels), observes the resulting changes in the classifier's output, and assigns importance scores to regions based on their influence on the predicted class. The decoder is not involved in this explanation process. It is used only after masking to reconstruct images for further evaluation. In this way, LIME effectively explains the classifier's behavior through localized perturbations in the image space while leveraging the latent representations generated by the VAE.

\vspace{1em}
\begin{algorithm}[H]
\caption{LIME on Images for Counterfactual Generation}
\label{alg:lime_on_images}
\begin{algorithmic}[1]
\REQUIRE Image $I$, encoder $E$, decoder $D$, classifier $C$, LIME explainer $L$
\ENSURE Whether a counterfactual explanation is found

\STATE Encode image: $z \leftarrow E(I)$
\STATE Predict label: $y_{\text{orig}} \leftarrow \arg\max C(z)$

\STATE Generate explanation: $M \leftarrow L(I, C)$
\STATE Select top-$k$ important regions: $R \leftarrow \text{Mask}(M, k=5)$
\STATE Apply alpha-blended mask: $I_{\text{masked}} \leftarrow \text{Blend}(I, R, \alpha=0.5)$

\STATE $z_{\text{masked}} \leftarrow E(I_{\text{masked}})$
\STATE $I_{\text{recon}} \leftarrow D(z_{\text{masked}})$
\STATE $z_{\text{re}} \leftarrow E(I_{\text{recon}})$
\STATE Predict new label: $y_{\text{new}} \leftarrow \arg\max C(z_{\text{re}})$

\IF{$y_{\text{new}} \neq y_{\text{orig}}$}
    \RETURN Counterfactual explanation found
\ENDIF

\RETURN No counterfactual explanation found
\end{algorithmic}
\end{algorithm}
\vspace{1em}

Unlike fixed grid-based masking, LIME adapts to each image by highlighting only the most relevant regions. This makes it more data-driven and context-aware. However, due to its reliance on local perturbations and repeated forward passes, it is comparatively more computationally intensive. Its interpretability makes it a valuable method for understanding model behavior in safety-critical domains such as autonomous driving.



\subsection{LIME-Based Masking on Latent Features} \label{sec:lime_based_masking_on_latent_features}

\subsubsection*{Latent Feature Statistics Preprocessing}
\label{sec:latent_statistics_preprocessing}
Before applying LIME-based masking in the latent space, it is essential to establish dataset-level statistical priors for each latent dimension. This ensures that when latent features are perturbed or replaced, the substituted values remain within a plausible range, preserving semantic consistency.

To compute these priors, all images from the training and test sets are encoded using the trained Variational Autoencoder (VAE) encoder, yielding a 128-dimensional latent vector for each image. The collection of these vectors across the dataset is used to compute per-dimension statistics, including the mean, median, minimum, maximum, and standard deviation.

These computed statistics are used during the masking phase to replace influential latent features. For instance, if LIME identifies a latent dimension as critical to the current classification, its value can be substituted with the median or mean value from the dataset to simulate a more neutral configuration. This strategy ensures that the modified latent vectors remain close to the data manifold, improving the realism of generated counterfactuals.

All individual latent vectors are stored as \texttt{.npy} files, and the computed statistics are saved as CSV files (\texttt{median\_values.csv}, \texttt{mean\_values.csv}, etc.) for efficient lookup during the masking phase. These statistics serve as the foundation for the LIME on Latent Features method and its Nearest Unchanged Neighbor (NUN) extension.

\subsection{LIME on Latent Features}
\label{sec:lime_on_latent}

This method applies the LIME (Local Interpretable Model-Agnostic Explanations) algorithm directly to the latent space of a trained Variational Autoencoder (VAE) to identify and mask the most influential latent dimensions contributing to the model's classification. Unlike image-space LIME, this approach explores the classifier’s decision boundary within the compressed latent representation of the input image.

Prior to this step, dataset-level statistics such as mean and median are computed across all latent dimensions, as detailed in Section~\ref{sec:latent_statistics_preprocessing}. These values serve as plausible replacements when masking latent features.

Given an input image, the VAE encoder generates a latent vector $z \in \mathbb{R}^{128}$, which is classified by a downstream neural classifier to obtain the original prediction. LIME is then used to identify the most positively influential latent dimensions. These selected features are masked iteratively by replacing them with their corresponding median values from the dataset.

After each masking step, the modified latent vector is passed through the VAE decoder to reconstruct the image. The reconstructed image is then re-encoded and re-classified to check for a prediction change. If the classification differs from the original, a counterfactual explanation (CE) is considered successfully generated.

\vspace{1em}
\begin{algorithm}[H]
\caption{LIME-Based Masking on Latent Features}
\label{alg:lime_on_latent}
\begin{algorithmic}[1]
\REQUIRE Image $I$, encoder $E$, decoder $D$, classifier $C$, median latent vector $\bar{z}$
\ENSURE Whether a counterfactual explanation is found

\STATE $z \leftarrow E(I)$ \hfill // Encode image to latent vector
\STATE $y_{\text{orig}} \leftarrow \arg\max C(z)$ \hfill // Original prediction

\STATE Use LIME to compute feature importance scores on $z$
\STATE Select positively influential features $\mathcal{F}$, sorted by importance
\FOR{feature index $i$ in $\mathcal{F}$}
    \STATE $z_{\text{masked}} \leftarrow z$ with $z_i \leftarrow \bar{z}_i$
    \STATE $I_{\text{recon}} \leftarrow D(z_{\text{masked}})$
    \STATE $z_{\text{re}} \leftarrow E(I_{\text{recon}})$
    \STATE $y_{\text{new}} \leftarrow \arg\max C(z_{\text{re}})$
    \IF{$y_{\text{new}} \neq y_{\text{orig}}$}
        \RETURN Counterfactual explanation found
    \ENDIF
\ENDFOR

\RETURN No counterfactual explanation found
\end{algorithmic}
\end{algorithm}
\vspace{1em}

This method provides a more abstract yet powerful mechanism for generating counterfactuals, as it operates on high-level features learned by the VAE. The use of dataset-informed median values ensures semantic plausibility, while early stopping upon the first successful prediction change improves efficiency. Visualizations of LIME feature importance and masked features are generated to enhance interpretability.






\subsection{LIME-Based Maksing on Latent Features using NUN} \label{lime_with_NUN}
In this work, we extend the counterfactual explanation framework introduced by Wijekoon et al.~\cite{WijekoonWNMPC21}, originally developed for tabular data, to the domain of deep generative models operating in image latent spaces. Our method applies LIME-based latent feature masking guided by a Nearest Unlike Neighbor (NUN) to generate visually meaningful and minimally modified counterfactuals.

Given a query image, we first encode it into a latent representation using a trained Variational Autoencoder (VAE). We then search the dataset for the nearest latent vector that belongs to a different class, i.e., the NUN, using Euclidean distance in the latent space. Once identified, we apply LIME (Local Interpretable Model-Agnostic Explanations) on the NUN’s latent vector to obtain a ranked list of latent features based on their influence on the classification outcome. This aligns with the second method proposed by Wijekoon et al.~\cite{WijekoonWNMPC21}, which prioritizes NUN-driven feature importance over query driven relevance for improved actionability.

We then iteratively replace the latent features of the query with their corresponding values from the NUN, in the order of their LIME-derived importance, and monitor the classifier output after each step. The process continues until a change in predicted class is observed, ensuring that the minimal number of feature modifications necessary to induce a decision flip are performed. The final modified latent vector is decoded to generate the counterfactual image, which is evaluated using both classifier confidence and image similarity metrics such as SSIM, PSNR, MSE, UQI, and VIFP.

This method effectively integrates explainability (via LIME) and semantic realism (via NUN), allowing us to generate actionable and interpretable counterfactuals in complex visual domains while adhering to the principle of minimality.

The detailed procedure for generating counterfactuals using NUN and LIME-based latent masking is summarized in Algorithm~\ref{alg:nun_lime}.


\begin{algorithm}[H]
\caption{LIME-Based Masking on Latent Features using NUN}
\label{alg:nun_lime}
\begin{algorithmic}[1]
\REQUIRE Encoder $E$, Decoder $D$, Classifier $C$, Query image $I$, Test dataset $\mathcal{D}_{\text{test}}$, Label mapping $\mathcal{L}$
\ENSURE Counterfactual image $x_{\text{cf}}$, Number of features replaced $k$, Evaluation metrics

\STATE $z \leftarrow E(I)$ \hfill \textit{// Encode query to latent space}
\STATE $y \leftarrow \arg\max C(z)$ \hfill \textit{// Get predicted label}
\STATE $z_{\text{NUN}} \leftarrow \text{None}$, $d_{\min} \leftarrow \infty$

\FORALL{$J \in \mathcal{D}_{\text{test}}$}
    \STATE $z_j \leftarrow E(J)$
    \STATE $y_j \leftarrow \arg\max C(z_j)$
    \IF{$y_j \ne y$}
        \STATE $d \leftarrow \|z - z_j\|_2$
        \IF{$d < d_{\min}$}
            \STATE $z_{\text{NUN}} \leftarrow z_j$
            \STATE $d_{\min} \leftarrow d$
        \ENDIF
    \ENDIF
\ENDFOR

\STATE $\text{importance} \leftarrow \text{LIME}(z_{\text{NUN}}, C)$
\STATE $\mathcal{F} \leftarrow \text{SortDescendingByImportance}(\text{importance})$
\STATE $z_{\text{mod}} \leftarrow z$, $k \leftarrow 0$

\FORALL{$i \in \mathcal{F}$}
    \STATE $z_{\text{mod}}[i] \leftarrow z_{\text{NUN}}[i]$
    \STATE $x_{\text{recon}} \leftarrow D(z_{\text{mod}})$
    \STATE $z_{\text{reenc}} \leftarrow E(x_{\text{recon}})$
    \STATE $y_{\text{new}} \leftarrow \arg\max C(z_{\text{reenc}})$
    \STATE $k \leftarrow k + 1$
    \IF{$y_{\text{new}} \ne y$}
        \STATE \textbf{break}
    \ENDIF
\ENDFOR

\STATE $\text{metrics} \leftarrow \text{Evaluate}(I, x_{\text{recon}})$
\RETURN $x_{\text{recon}}, k, \text{metrics}$
\end{algorithmic}
\end{algorithm}








\section{Generating Counterfactual Explanations}
Process of modifying inputs to generate counterfactuals.

\subsection{Counterfactual Explanation Generation Pipeline}
Step-by-step process of generating counterfactuals.


\subsection{Example of Counterfactual Generation}
a visual example of an image before and after counterfactual modification.

\chapter{Evaluation and Results} \label{Evaluation and Results}
As introduced in \cref{Introduction}, this thesis explores the generation of counterfactual explanations (CEs) in image-based autonomous driving scenarios using deep generative models and post-hoc interpretability techniques. The corresponding implementations are detailed in \cref{Methodology}.

This chapter presents the evaluation strategy and experimental findings, structured according to the central research questions (RQs). For each RQ, the evaluation criteria are outlined and later expanded in dedicated result sections.

\textbf{RQ1:} \textit{How can deep generative models be implemented to effectively encode and reconstruct high-dimensional RGB input images of driving scenes for compact and interpretable representation learning in autonomous driving systems?}

\vspace{-1em}

\paragraph{Evaluation:}The VAE is evaluated on two primary fronts: (i) the fidelity of reconstructed images, and (ii) the semantic usefulness of the latent representations for downstream classification. To this end, multiple classifiers—including MLP, SVM, KNN, Logistic Regression, and Random Forest—are trained on the latent vectors extracted by the encoder. Classification performance is assessed for both binary (GO vs. STOP) and multi-class (GO, STOP, LEFT, RIGHT) tasks. Evaluation metrics include reconstruction loss, SSIM, PSNR, classification accuracy, precision, recall, and F1-score.
    
\vspace{1em}

\textbf{RQ2:} \textit{How can loss function modifications in Variational Autoencoders (VAEs) optimize image reconstruction quality in autonomous driving tasks?} 

\vspace{-1em}

\paragraph{Evaluation:}Compare the image quality Structural Similarity Index (SSIM) and Peak Signal-to-Noise Ratio (PSNR) and classification metrics (accuracy, F1) between a standard VAE and one using modified loss functions (logcosh loss). This will include both quantitative analysis and qualitative visualizations of reconstructed images.

\vspace{1em}
    
\textbf{RQ3:} \textit{How do different masking techniques impact the effectiveness and efficiency of counterfactual explanation generation, in terms of coverage, computational cost, method overlap, and failure rate?}

\vspace{-1em}



\paragraph{Evaluation:} To compare the five counterfactual generation methods, we use the following evaluation dimensions:
    \begin{itemize} 
        \item \textbf{Coverage:} The percentage of input samples for which a valid counterfactual explanation is found. A valid CE is one that causes a change in the classifier's prediction after masking and reconstruction.
        
        \item \textbf{Failure Rate:} The proportion of inputs for which the method fails to generate a counterfactual. This includes cases where the masked reconstruction does not lead to a prediction flip.
    
        \item \textbf{Computational Cost:} The average time taken (in seconds) to generate a counterfactual for each input. This reflects the practical usability and scalability of the masking strategy.
    
        \item \textbf{Method Overlap:} The number and percentage of cases where multiple methods yield the same counterfactual outcome (i.e., same final predicted label after masking). This metric helps evaluate the redundancy or uniqueness of each technique.

    \end{itemize}

    All masking methods are applied to the same dataset split, using identical model checkpoints for the VAE and classifier. This ensures fair and consistent comparisons across metrics.

\vspace{1em}

\textbf{RQ4:} \textit{ Which counterfactual explanation method is preferred by users when selecting among generated explanations of the same original image? What factors influence user preference? (See \cref{Methodology} for details on the methods compared.)}

\vspace{-1em}

\paragraph{Evaluation:} Evaluation is based on aggregated user responses:
    \begin{itemize}
        \item \textbf{Preference Distribution:} Percentage of total selections made per CE method.
        \item \textbf{Attribute Ratings:} Average scores for interpretability, plausibility, and coherence across methods.
        \item \textbf{Qualitative Feedback:} Comments collected through optional text fields to gain insights into user reasoning.
    \end{itemize}

This dual analysis enables the identification of methods that are not only algorithmically effective but also intuitively meaningful to end users critical for real-world deployment of interpretable AI in autonomous systems.
    


Each of the following sections restates the respective RQ, evaluates the implemented methods against that question, and presents results along with a discussion of key findings.


\section{Evaluation of Variational Autoencoder (RQ1, RQ2)} \label{sec:vae_evaluation}

The Variational Autoencoder (VAE) forms the backbone of this thesis. It serves dual purposes: (i) as a deep generative model that enables the reconstruction and manipulation of high-dimensional driving scene images, and (ii) as a representation learning tool to generate low-dimensional, semantically meaningful latent vectors suitable for downstream driving action classification. The quality of both image reconstruction and latent encoding is crucial for the success of counterfactual generation. Therefore, evaluating the VAE both quantitatively and qualitatively is a vital first step.

To ensure robustness, various architectural configurations and hyperparameter settings were explored, including latent dimensionality, convolutional depth, and KL divergence annealing schedules. One major focus was the choice of reconstruction loss function, which has a significant effect on the quality of output and the stability of training. Two variants were trained. One using traditional Mean Squared Error (MSE) and the other using Log-Cosh, a smoother, outlier-robust alternative motivated by prior work by Chen et al.,~\cite{chen2019log}. This section presents the training dynamics, comparative results between the two losses, and the final analysis of latent space quality.

\subsubsection{Latent Space Selection and Architectural Rationale} \label{sec:latent-space-selection}

An empirical comparison of latent dimensionalities 64, and 128 was conducted to determine the optimal balance between reconstruction fidelity, training stability, and latent representation capacity. All other architectural and training parameters were held constant for a fair comparison.
\begin{itemize}
    \item \textbf{64-Dimensional:} Produced significantly blurry reconstructions refer figure~\ref{fig:vae64_recon}, particularly around edges and textures. While training remained stable, the compressed latent space struggled to retain sufficient visual detail.
    
    \item \textbf{128-Dimensional:} Achieved the best balance — sharp reconstructions refer figure~\ref{fig:vae128_recon}, stable training, and latent vectors that, when visualized using PCA and t-SNE, showed class-specific clustering post hoc see figure~\ref{fig:latent_space_visualizations}.
\end{itemize}

\begin{figure}[htbp]
    \centering
    \begin{subfigure}[b]{0.6\textwidth}
        \includegraphics[width=\textwidth]{img/vae_results/epoch_200.png}
        \caption{VAE reconstruction with 64-D latent space}
        \label{fig:vae64_recon}
    \end{subfigure}
    \hfill
    \begin{subfigure}[b]{0.6\textwidth}
        \includegraphics[width=\textwidth]{img/vae_results/200_epochs_128_ls_logcosh/reconstructions/epoch_200.png}
        \caption{VAE reconstruction with 128-D latent space}
        \label{fig:vae128_recon}
    \end{subfigure}

    \caption[Reconstruction comparison for different VAE latent dimensions]{%
Qualitative comparison of reconstructed RGB images using different latent dimensionalities. The 64-dimensional model shows noticeable blurring and structural loss, while the 128-dimensional variant preserves edges and semantic layout more effectively.}
    \label{fig:vae_latent_qual_comparison}
\end{figure}

\begin{table}[!h]
    \centering
    \caption{Comparison of VAE latent dimensionalities.}
    \label{tab:vae-latent-comparison}
    \begin{tabular}{|c|p{4cm}|p{3.5cm}|p{4cm}|}
        \hline
        \textbf{Latent Dim} & \textbf{Reconstruction Quality} & \textbf{Training Stability} & \textbf{Latent Structure (t-SNE)} \\
        \hline
        64 & Blurry edges, missing details & Stable & Weak structure \\
        128 & Clear, high-fidelity images & Stable & Moderate clustering \\
        \hline
    \end{tabular}
\end{table}

The final VAE architecture design motivated from Idrees Shaikh’s CARLA DRL project\footnote{\url{https://github.com/idreesshaikh/Autonomous-Driving-in-Carla-using-Deep-Reinforcement-Learning}}, which originally employed semantic segmentation and a 64-dimensional latent space. Since this thesis works directly with raw RGB images, preserving color and fine-grained spatial detail required a richer representation. Accordingly, the latent dimension was increased to 128, and the architecture was fine-tuned through multiple iterations to ensure consistent convergence and high-quality reconstructions.

\clearpage



\subsection{Training Performance and Loss Analysis} \label{subsubsec:vae_training_loss}

Two VAE variants were trained for 200 epochs each (see hyperparameters in \cref{subsec:hyperparameter_config}), differing only in their reconstruction loss function: one using Mean Squared Error (MSE) and the other using Log-Cosh. Throughout training, several key performance metrics were monitored to evaluate model convergence, reconstruction quality, and the behavior of latent space regularization.

The \textbf{KL divergence} reflects how well the learned latent distribution aligns with the unit Gaussian prior, and its formulation is rooted in the VAE loss function discussed in \cref{subsec:VAE Objective Function and Loss} and \cref{subsec:reparameterization_trick}. The \textbf{reconstruction loss} measures pixel-wise error between original and generated images, directly reflecting the decoder's effectiveness and forming one half of the ELBO objective, as described in \cref{subsec:VAE Objective Function and Loss}. The \textbf{total loss}, combining reconstruction loss and KL divergence (with an annealing weight), serves as the main optimization target and is detailed in the ELBO formulation. 

In addition to these theoretical components, perceptual metrics were employed to evaluate the quality of generated images. The \textbf{Structural Similarity Index (SSIM)} captures human-perceived image similarity and complements traditional pixel-based losses. Similarly, the \textbf{Peak Signal-to-Noise Ratio (PSNR)} offers a quantitative measure of image fidelity. Finally, \textbf{pixel accuracy} assesses how closely the reconstructed image matches the original input on a pixel-wise level. These perceptual evaluations provide practical insights that go beyond the theoretical loss terms presented in \cref{sec:variational_inference}.

Figure~\ref{fig:vae_loss_comparison_1} and Figure~\ref{fig:vae_loss_comparison_2} provide a side-by-side comparison of these key metrics for both VAE models trained with Log-Cosh and MSE losses.

\begin{figure}[p]
    \centering
    % --- KL Divergence
    \begin{subfigure}[b]{0.48\textwidth}
        \includegraphics[width=\textwidth]{img/vae_results/200_epochs_128_ls_logcosh/logcosh_kl_loss.png}
        \caption{KL Divergence — Log-Cosh}
    \end{subfigure}
    \hfill
    \begin{subfigure}[b]{0.48\textwidth}
        \includegraphics[width=\textwidth]{img/vae_results/200_epochs_128_ls_mse/mse_kl_loss.png}
        \caption{KL Divergence — MSE}
    \end{subfigure}

    % --- Reconstruction Loss
    \begin{subfigure}[b]{0.48\textwidth}
        \includegraphics[width=\textwidth]{img/vae_results/200_epochs_128_ls_logcosh/logcosh_recon_loss.png}
        \caption{Reconstruction Loss — Log-Cosh}
    \end{subfigure}
    \hfill
    \begin{subfigure}[b]{0.48\textwidth}
        \includegraphics[width=\textwidth]{img/vae_results/200_epochs_128_ls_mse/mse_recon_loss.png}
        \caption{Reconstruction Loss — MSE}
    \end{subfigure}

    % --- Total Loss
    \begin{subfigure}[b]{0.48\textwidth}
        \includegraphics[width=\textwidth]{img/vae_results/200_epochs_128_ls_logcosh/logcosh_total_loss.png}
        \caption{Total Loss — Log-Cosh}
    \end{subfigure}
    \hfill
    \begin{subfigure}[b]{0.48\textwidth}
        \includegraphics[width=\textwidth]{img/vae_results/200_epochs_128_ls_mse/mse_total_loss.png}
        \caption{Total Loss — MSE}
    \end{subfigure}

    \caption[Loss comparison for Log-Cosh and MSE-trained VAEs]{%
Comparison of KL divergence, reconstruction loss, and total loss across training epochs for VAE models trained with Log-Cosh and MSE losses. The Log-Cosh model shows more stable convergence with better structural recovery.}
    \label{fig:vae_loss_comparison_1}
\end{figure}


\begin{figure}[p]
    \centering
    % --- Pixel Accuracy
    \begin{subfigure}[b]{0.48\textwidth}
        \includegraphics[width=\textwidth]{img/vae_results/200_epochs_128_ls_logcosh/logcosh_val_accuracy.png}
        \caption{Validation Accuracy — Log-Cosh}
    \end{subfigure}
    \hfill
    \begin{subfigure}[b]{0.48\textwidth}
        \includegraphics[width=\textwidth]{img/vae_results/200_epochs_128_ls_mse/mse_val_accuracy.png}
        \caption{Validation Accuracy — MSE}
    \end{subfigure}

    % --- PSNR
    \begin{subfigure}[b]{0.48\textwidth}
        \includegraphics[width=\textwidth]{img/vae_results/200_epochs_128_ls_logcosh/logcosh_val_psnr.png}
        \caption{PSNR — Log-Cosh}
    \end{subfigure}
    \hfill
    \begin{subfigure}[b]{0.48\textwidth}
        \includegraphics[width=\textwidth]{img/vae_results/200_epochs_128_ls_mse/mse_val_psnr.png}
        \caption{PSNR — MSE}
    \end{subfigure}

    % --- SSIM
    \begin{subfigure}[b]{0.48\textwidth}
        \includegraphics[width=\textwidth]{img/vae_results/200_epochs_128_ls_logcosh/logcosh_val_ssim.png}
        \caption{SSIM — Log-Cosh}
    \end{subfigure}
    \hfill
    \begin{subfigure}[b]{0.48\textwidth}
        \includegraphics[width=\textwidth]{img/vae_results/200_epochs_128_ls_mse/mse_val_ssim.png}
        \caption{SSIM — MSE}
    \end{subfigure}

    \caption[Performance metrics for Log-Cosh vs MSE VAEs]{%
Evaluation metrics for VAEs trained with Log-Cosh and MSE loss functions. Metrics include validation accuracy, PSNR (Peak Signal-to-Noise Ratio), and SSIM (Structural Similarity Index), showcasing better reconstruction fidelity for the Log-Cosh variant.}
    \label{fig:vae_loss_comparison_2}
\end{figure}

\clearpage

The comparative plots in Figure~\ref{fig:vae_loss_comparison_1} and Figure~\ref{fig:vae_loss_comparison_2} clearly demonstrate that the VAE trained with the Log-Cosh loss outperforms its MSE counterpart across all major evaluation criteria:

\begin{itemize}
    \item \textbf{KL Divergence:} Both models show stable convergence, but the Log-Cosh variant achieves a marginally lower KL loss by the end of training, indicating more effective regularization of the latent space (Figures~\ref{fig:vae_loss_comparison_1}a and~\ref{fig:vae_loss_comparison_1}b).
    
    \item \textbf{Reconstruction and Total Loss:} The Log-Cosh model exhibits faster convergence and lower final loss values, suggesting improved robustness during training (Figures~\ref{fig:vae_loss_comparison_1}c to~\ref{fig:vae_loss_comparison_1}f).
    
    \item \textbf{Pixel Accuracy:} As shown in Figures~\ref{fig:vae_loss_comparison_2}a and~\ref{fig:vae_loss_comparison_2}b, the Log-Cosh model consistently yields higher pixel accuracy, peaking at approximately 96.96\%.
    
    \item \textbf{PSNR and SSIM:} The perceptual quality of reconstructions is higher with Log-Cosh, achieving 31.04 dB PSNR and 0.864 SSIM (Figures~\ref{fig:vae_loss_comparison_2}c to~\ref{fig:vae_loss_comparison_2}f), compared to 29.44 dB and 0.823 with MSE.
\end{itemize}

Visual inspection further confirms these findings. Figure~\ref{fig:vae_qualitative_comparison} presents side-by-side reconstructions at epoch 200. The Log-Cosh model preserves finer details, textures, and structural integrity, whereas the MSE-based reconstruction exhibits noticeable blurring, particularly around object boundaries, disappearing of the traffic pole and fine features.

\begin{figure}[htbp]
    \centering
    \begin{subfigure}[b]{0.45\textwidth}
        \includegraphics[width=\textwidth]{img/vae_results/200_epochs_128_ls_logcosh/reconstructions/epoch_200.png}
        \caption{Log-Cosh — Epoch 200}
        \label{fig:logcosh_epoch200}
    \end{subfigure}
    \hfill
    \begin{subfigure}[b]{0.45\textwidth}
        \includegraphics[width=\textwidth]{img/vae_results/200_epochs_128_ls_mse/reconstructions/epoch_200.png}
        \caption{MSE — Epoch 200}
        \label{fig:mse_epoch200}
    \end{subfigure}
    \caption[Qualitative comparison of VAE reconstructions (Log-Cosh vs. MSE)]{%
Side-by-side qualitative comparison of VAE reconstructions at epoch 200. The Log-Cosh model (left) yields sharper and more semantically consistent outputs than the MSE model (right).}
    \label{fig:vae_qualitative_comparison}
\end{figure}







\clearpage


\subsection{Quantitative Comparison of Loss Functions (RQ2 Answered)} \label{subsec:vae_quant_comparison}
To assess the performance of the Variational Autoencoder (VAE), both the Log-Cosh and Mean Squared Error (MSE) loss functions were evaluated across several key metrics: Total Loss, Reconstruction Loss, KL Divergence, Peak Signal-to-Noise Ratio (PSNR), and Structural Similarity Index (SSIM). The Total Loss is computed as the sum of the reconstruction loss (Log-Cosh or MSE) and the weighted Kullback-Leibler (KL) divergence, where the KL term regularizes the latent space and its contribution is gradually increased during training using KL annealing.

\begin{equation}
\mathcal{L}_{\text{VAE}} = \mathcal{L}_{\text{recon}} + \lambda_{\text{KL}} \cdot \mathcal{L}_{\text{KL}}
\end{equation}

Here, $\mathcal{L}_{\text{recon}}$ denotes the reconstruction loss (either Log-Cosh or MSE), $\mathcal{L}_{\text{KL}}$ is the KL divergence between the approximate posterior and the standard normal prior, and $\beta$ is a dynamic weight term that increases gradually during training to balance reconstruction fidelity with latent space regularization.

Table~\ref{tab:loss_comparison} presents average values over the final 5 epochs for both variants:

\begin{table}[h]
    \centering
    \begin{tabular}{lcc}
        \toprule
        \textbf{Metric} & \textbf{Log-Cosh (Avg)} & \textbf{MSE (Avg)} \\
        \midrule
        Validation Total Loss & 36.13 & 62.21 \\
        Validation Reconstruction Loss & 30.77 & 55.63 \\
        Validation KL Divergence & 270.11 & 331.46 \\
        Validation PSNR (dB) & 29.43 & 28.80 \\
        Validation SSIM & 0.8230 & 0.8088 \\
        \bottomrule
    \end{tabular}
    \caption{Comparison of validation metrics for VAE variants (averaged over final 5 epochs).}
    \label{tab:loss_comparison}
\end{table}

Figures~\ref{fig:val_loss_comparison}, \ref{fig:val_psnr_comparison}, and \ref{fig:val_ssim_comparison} provide a visual comparison of how both losses influence model behavior throughout training.

\begin{figure}[h]
    \centering
    \includegraphics[width=0.7\textwidth]{img/vae_results/val_loss_comparison.png}
    \caption{Validation Loss Comparison: Log-Cosh vs. MSE over all epochs.}
    \label{fig:val_loss_comparison}
\end{figure}

\begin{figure}[h]
    \centering
    \includegraphics[width=0.7\textwidth]{img/vae_results/val_psnr_comparison.png}
    \caption{Validation PSNR Comparison: Log-Cosh vs. MSE over all epochs.}
    \label{fig:val_psnr_comparison}
\end{figure}

\begin{figure}[h]
    \centering
    \includegraphics[width=0.7\textwidth]{img/vae_results/val_ssim_comparison.png}
    \caption{Validation SSIM Comparison: Log-Cosh vs. MSE over all epochs.}
    \label{fig:val_ssim_comparison}
\end{figure}

From these results:

\begin{itemize}
    \item \textbf{Figure~\ref{fig:val_loss_comparison}} shows that the Log-Cosh loss function results in faster and more stable convergence compared to MSE. After the initial 20 epochs, the loss curve for Log-Cosh remains consistently lower, confirming its robustness against outliers and noisy gradients.

    \item \textbf{Figure~\ref{fig:val_psnr_comparison}} demonstrates that the Log-Cosh variant yields higher PSNR throughout training. Since PSNR is a direct indicator of reconstruction quality, this suggests that reconstructions from the Log-Cosh model are closer to the original inputs in pixel space.

    \item \textbf{Figure~\ref{fig:val_ssim_comparison}} provides further evidence that the Log-Cosh loss leads to more perceptually realistic outputs. The SSIM curve, which accounts for luminance, contrast, and structural similarity, consistently favors Log-Cosh, particularly in the later epochs, indicating better structural retention in the reconstructions which is clearly shown in the plot blue line indicated the log-cosh loss ssim and orange indicates the mse loss ssim.

    \item Additionally, the KL divergence is lower in the Log-Cosh variant (Table~\ref{tab:loss_comparison}), indicating a better trade-off between regularization and reconstruction fidelity.
\end{itemize}

Overall, these findings confirm that the Log-Cosh loss not only provides better quantitative performance but also leads to reconstructions that are perceptually more accurate and visually coherent. This clearly answers RQ2 by validating Log-Cosh as the superior choice for training VAEs in this context.


\clearpage


\subsection{Visual Comparison of Reconstructions} \label{subsubsec:vae_visual_recon}
Representative input images and corresponding reconstructions from both models at epoch 200 were visually compared.

Log-Cosh Reconstructions (Figure~\ref{fig:logcosh_epoch200}) retain road boundaries, tree textures, and lighting gradients with greater smoothness and fewer artifacts.

MSE Reconstructions (Figure~\ref{fig:mse_epoch200}) exhibit noise in edge regions and less coherent texture patterns and lost many important details like traffic light pole, road boundary and many more.

These qualitative results further confirms that adopting the Log-Cosh loss, as motivated by Chen et al.~\cite{chen2019log}, substantially improves VAE performance thereby answering RQ2 (How can loss function modifications in Variational Autoencoders (VAEs) optimize image reconstruction quality in autonomous driving tasks?).


%\vspace{1em}
\subsection{Latent Space Visualization and Separability}
\label{subsubsec:vae_latent_space}

Although the VAE in this thesis is trained in an unsupervised manner without using class labels, one of its key objectives is to learn a latent space that not only enables accurate image reconstruction but also supports downstream classification and interpretability. To examine this, the latent vectors produced by the encoder were evaluated using a separate classifier trained on top of these vectors. Despite the lack of supervision during VAE training, the classifier demonstrated strong performance, indicating that the learned latent representations capture semantically meaningful and class-relevant features.

To qualitatively assess the structure of the latent space, dimensionality reduction techniques were applied to the 128-dimensional latent vectors extracted from the test set. These vectors were projected into two dimensions using Principal Component Analysis (PCA) and t-Distributed Stochastic Neighbor Embedding (t-SNE). The resulting embeddings are shown in Figure~\ref{fig:latent_space_visualizations}, where each point is colored according to its ground-truth class label.

\begin{figure}[htbp]
    \centering
    \begin{subfigure}[b]{0.49\textwidth}
        \includegraphics[width=\textwidth]{img/vae_results/media_images_PCA Latent_0_1f2a94b6feecab3585be.png}
        \caption{PCA projection of latent vectors (by true labels).}
        \label{fig:pca_true}
    \end{subfigure}
    \hfill
    \begin{subfigure}[b]{0.49\textwidth}
        \includegraphics[width=\textwidth]{img/vae_results/media_images_t-SNE Latent_1_77cd107489e31c6c9f4d.png}
        \caption{t-SNE projection of latent vectors (by true labels).}
        \label{fig:tsne_true}
    \end{subfigure}
    \caption[Latent space projections using PCA and t-SNE]{%
Latent space projections using PCA and t-SNE for test set samples. Each point represents an encoded image projected to 2D. Colors denote ground-truth labels: Class 0 = \texttt{STOP}, Class 1 = \texttt{GO}, Class 2 = \texttt{LEFT}, Class 3 = \texttt{RIGHT}.}
    \label{fig:latent_space_visualizations}
\end{figure}


These visualizations illustrate the extent to which semantically similar driving scenes are grouped in the latent space. While some local clustering is visible—particularly for the \texttt{STOP} (Class 0) and \texttt{RIGHT} (Class 3) classes—the overall class separability is limited. Notably, the \texttt{GO} (Class 1) and \texttt{LEFT} (Class 2) classes show significant overlap, which may reflect their visual similarity in real-world scenarios such as urban intersections.

This observation reflects a core tradeoff in VAE training: while the KL divergence term encourages smooth and continuous latent distributions, it can also limit the formation of sharply separable class clusters—especially when such clusters do not emerge naturally from the reconstruction objective alone.

Importantly, since the VAE was trained entirely without labels, the latent space structure observed here arises organically from its goal of reconstructing input images while conforming to a Gaussian prior. As such, latent space visualizations and separability metrics serve as useful tools for evaluating the VAE’s representational capacity and its suitability for downstream tasks, including classification and counterfactual explanation generation discussed in later sections.












\clearpage



\section{Evaluation of Classifiers Trained on Latent Features} \label{sec:classifier_eval}

Following the successful design and training of the Variational Autoencoder (VAE), the next step in this thesis was to assess the effectiveness of its learned latent space in enabling downstream decision-making. Specifically, this section evaluates whether the encoded 128-dimensional latent representations can support accurate multi-class classification of driving actions. This directly addresses the core objective of \textbf{RQ1}, by investigating the semantic richness and discriminative power of the latent vectors produced by the VAE.

To this end, a set of traditional and neural classifiers were trained and evaluated on the same VAE-derived latent representations, using a balanced four-class dataset comprising \texttt{STOP}, \texttt{GO}, \texttt{LEFT}, and \texttt{RIGHT} labels. Five models were studied: Logistic Regression (baseline linear classifier), K-Nearest Neighbors (KNN), Random Forest (RF), Support Vector Machine (SVM), and Neural Network (NN).

Classifier training implementation details are described in \cref{sec:classifier_architectures}. Performance evaluation was carried out using a comprehensive set of metrics such as accuracy, per-class precision, recall, F1-score, macro and weighted averages, confusion matrices, and ROC-AUC curves for one-vs-rest classification~\cite{labelf2025metrics, roc_auc2024, svm_rf_knn2017}.



\subsection{Evaluations and Results of Traditional Classifiers} \label{subsec:comparision_with_traditional_classifiers}

To contextualize the performance of the neural model, four traditional classifiers were evaluated using the same latent vectors. Training and implementations of these classifiers are elaborated in \cref{sec:classifier_architectures}. Figure~\ref{fig:comparison_matrices} summarizes the confusion matrices for each model, and Figure~\ref{fig:comparison_rocs} provides the corresponding ROC curves.

\begin{figure}[h]
\centering
\begin{subfigure}{0.49\textwidth}
    \includegraphics[width=\linewidth]{img/classifier/logistic_regression_confucion_matrix.png}
    \caption{Logistic Regression}
\end{subfigure}
\begin{subfigure}{0.49\textwidth}
    \includegraphics[width=\linewidth]{img/classifier/KNN_confucion_matrix.png}
    \caption{KNN}
\end{subfigure}

\vspace{0.5em}

\begin{subfigure}{0.49\textwidth}
    \includegraphics[width=\linewidth]{img/classifier/random_forest_confucion_matrix.png}
    \caption{Random Forest}
\end{subfigure}
\begin{subfigure}{0.49\textwidth}
    \includegraphics[width=\linewidth]{img/classifier/SVM_confucion_matrix.png}
    \caption{SVM}
\end{subfigure}

\caption{Confusion matrices for traditional classifiers.}
\label{fig:comparison_matrices}
\end{figure}


\begin{figure}[h]
\centering
\begin{subfigure}{0.48\textwidth}
    \includegraphics[width=\linewidth]{img/classifier/logistic_regression_AUC.png}
    \caption{Logistic Regression}
\end{subfigure}
\begin{subfigure}{0.48\textwidth}
    \includegraphics[width=\linewidth]{img/classifier/KNN_AUC.png}
    \caption{KNN}
\end{subfigure}

\vspace{0.5em}

\begin{subfigure}{0.48\textwidth}
    \includegraphics[width=\linewidth]{img/classifier/random_forest_AUC.png}
    \caption{Random Forest}
\end{subfigure}
\begin{subfigure}{0.48\textwidth}
    \includegraphics[width=\linewidth]{img/classifier/SVM_AUC.png}
    \caption{SVM}
\end{subfigure}

\caption{ROC curves and AUC values for traditional classifiers (1-vs-rest).}
\label{fig:comparison_rocs}
\end{figure}





\paragraph{Logistic Regression.}

The Logistic Regression classifier achieved an overall accuracy of 80\% and a macro F1-score of 0.80. As seen in Figure~\ref{fig:comparison_matrices}(a), the confusion matrix reveals that the model performs well for the \texttt{STOP} and \texttt{LEFT} classes, while showing significant misclassification for the \texttt{GO} and \texttt{RIGHT} classes. Notably, many \texttt{GO} instances were misclassified as \texttt{RIGHT} or \texttt{LEFT}, highlighting the difficulty in linearly separating these features in the latent space.

The ROC curves in Figure~\ref{fig:comparison_rocs}(a) further support this observation. While the \texttt{STOP} and \texttt{LEFT} classes achieved high AUC scores (0.98 and 0.96 respectively), the \texttt{GO} class recorded the lowest AUC at 0.88, indicating a higher overlap with other classes in the latent space.

Overall, Logistic Regression acts as a valuable baseline. Its performance confirms the need for non-linear classifiers to fully leverage the representational power of the VAE's latent space.

\paragraph{K-Nearest Neighbors (KNN).}

The K-Nearest Neighbors classifier achieved an overall accuracy of 88\% and a macro F1-score of 0.88, demonstrating competitive performance on the latent feature space extracted by the VAE. As shown in Figure~\ref{fig:comparison_matrices}(b), the model achieved near-perfect recall for the \texttt{STOP} class (99\%) and high precision for both \texttt{STOP} and \texttt{RIGHT}.

The most significant confusion occurred between the \texttt{LEFT} and \texttt{GO} classes, where 92 instances of \texttt{LEFT} were incorrectly predicted as \texttt{GO}, likely due to local similarity in features. This aligns with the proximity-based nature of KNN, which is sensitive to cluster overlap in high-dimensional spaces.

Figure~\ref{fig:comparison_rocs}(b) shows the ROC curves, where all classes achieve AUC values above 0.90. \texttt{STOP} reached the highest AUC of 0.99, followed by \texttt{LEFT} (0.95), \texttt{RIGHT} (0.94), and \texttt{GO} (0.91).

In summary, KNN offers a strong non-parametric baseline that effectively captures local structure in the VAE's latent space, though it may require further tuning or metric learning to fully disambiguate overlapping action classes.


\paragraph{Random Forest.}
The Random Forest classifier achieved an accuracy of 88\% and a macro F1-score of 0.88. As shown in Figure~\ref{fig:comparison_matrices}(c), the confusion matrix indicates strong performance across most classes, particularly for \texttt{STOP} (F1 = 0.95) and \texttt{RIGHT} (F1 = 0.89). The model also demonstrated competitive recall for the \texttt{GO} class (0.87), which often presents the most overlap with other categories.

ROC analysis (Figure~\ref{fig:comparison_rocs}(c)) revealed high AUC scores across the board, with \texttt{STOP} achieving a perfect AUC of 1.00, and the other classes maintaining AUCs above 0.95. These findings affirm the Random Forest’s ability to exploit the structure of the latent space effectively.

Complete classification metrics for all four classes are provided in Appendix~\ref{tab:random_forest_classification_report}, which reinforce the model’s balanced precision and recall across classes.

\paragraph{Support Vector Machine (SVM).}
The SVM classifier achieved an overall accuracy of 89\% and a macro F1-score of 0.89, outperforming several traditional models. As seen in Figure~\ref{fig:comparison_matrices}(d), the confusion matrix reveals strong and consistent performance across all classes. The model performed especially well for \texttt{STOP} and \texttt{LEFT} classes with F1-scores of 0.93 and 0.92 respectively, and demonstrated solid performance for \texttt{RIGHT} (F1 = 0.90). While the \texttt{GO} class had slightly lower scores (F1 = 0.81), it still surpassed performance seen in Logistic Regression and KNN.

The ROC curves in Figure~\ref{fig:comparison_rocs}(d) further support this, showing high AUC values across all classes, each reaching or exceeding 0.94. This indicates excellent class separability in the latent space when using a margin-based classifier with non-linear kernels.

A detailed breakdown of the classification metrics is available in Appendix~\ref{tab:svm_classification_report}, demonstrating that SVMs can leverage the non-linear separability embedded in the latent representations learned by the VAE.




Table~\ref{tab:classifier_comparison} presents a side-by-side comparison of accuracy and F1-scores across all classifiers:

\begin{table}[htbp]
\centering
\small
\begin{tabular}{p{3cm}ccp{5.5cm}}
\toprule
\textbf{Classifier} & \textbf{Accuracy} & \textbf{Macro F1} & \textbf{Notes} \\
\midrule
Logistic Regression & 80\% & 0.80 & Baseline linear classifier; struggled with \texttt{GO} \\
KNN & 88\% & 0.88 & High recall for \texttt{STOP}; sensitive to latent proximity \\
SVM & 89\% & 0.89 & Excellent class separability; strong on \texttt{LEFT} and \texttt{STOP} \\
Random Forest & 88\% & 0.88 & Robust performance; most errors in \texttt{GO} class \\
Neural Network & \textbf{89\%} & \textbf{0.89} & Best overall F1 balance; handles non-linear separations \\
\bottomrule
\end{tabular}
\caption{Performance comparison of classifiers trained on VAE latent features (4-class).}
\label{tab:classifier_comparison}
\end{table}


Across all classifiers, the \texttt{GO} class consistently had the lowest precision and recall. This suggests latent encodings for \texttt{GO} overlap more with other classes, possibly due to its transitional visual nature. Addressing this may require additional context, or spatial features.


\subsection{Evaluation of Neural Network Classifier (MLP)}
\label{subsec:classifier_trained_on_latentfeatures}

The neural network classifier, implemented as a Multi-Layer Perceptron (MLP), was trained for 20 epochs on the 128-dimensional latent features extracted from the VAE encoder. This model aimed to capture the non-linear relationships within the latent space and serve as a powerful baseline for downstream classification tasks.

Training progression is visualized in Figure~\ref{fig:loss_accuracy_plot}, which shows the learning dynamics over the course of training. The left plot demonstrates a consistent reduction in loss, while the right plot shows a steady increase in training accuracy, reaching a final value of approximately 93\%. These curves indicate effective convergence without overfitting, affirming the model's ability to generalize well on latent representations.

\begin{figure}[h]
    \centering
    \includegraphics[width=\textwidth]{img/classifier/training_loss_accuracy_4_classes.png}
    \caption[Training performance of 4-class neural classifier]{%
Training loss and accuracy curves for the neural classifier across 20 epochs. The plot illustrates the model's convergence and learning dynamics during training.}
    \label{fig:loss_accuracy_plot}
\end{figure}

Quantitative evaluation results on the test set are presented in Table~\ref{tab:classification_report}. The MLP achieved an overall accuracy of 89\% and a macro-averaged F1-score of 0.89, demonstrating robust performance across all four driving behavior classes.

\begin{table}[h]
    \centering
    \begin{tabular}{lcccc}
        \toprule
        \textbf{Class} & \textbf{Precision} & \textbf{Recall} & \textbf{F1-score} & \textbf{Support} \\
        \midrule
        STOP & 0.91 & 0.99 & 0.95 & 821 \\
        GO & 0.81 & 0.84 & 0.83 & 821 \\
        RIGHT & 0.93 & 0.86 & 0.89 & 821 \\
        LEFT & 0.93 & 0.89 & 0.91 & 821 \\
        \midrule
        \textbf{Accuracy} & \multicolumn{4}{c}{0.89 (3284 samples)} \\
        \textbf{Macro avg} & 0.90 & 0.89 & 0.89 & -- \\
        \textbf{Weighted avg} & 0.90 & 0.89 & 0.89 & -- \\
        \bottomrule
    \end{tabular}
    \caption{Classification report for the neural network trained on latent features (4-class).}
    \label{tab:classification_report}
\end{table}

The confusion matrix in Figure~\ref{fig:conf_matrix} provides additional insights into the model's performance by revealing class-specific error patterns. The \texttt{STOP} class is classified with the highest precision and recall, while the \texttt{GO} class remains the most challenging due to its visual and contextual overlap with other classes.

\begin{figure}[h]
    \centering
    \includegraphics[width=0.6\textwidth]{img/classifier/confusion_matrix_4_classes.png}
    \caption[Confusion matrix of neural classifier (4-class)]{%
Confusion matrix of the neural classifier on the 4-class test set. The matrix highlights per-class prediction performance and any misclassification trends.}
    \label{fig:conf_matrix}
\end{figure}

\begin{itemize}
    \item \textbf{STOP:} Most accurately classified class (Recall = 0.99), confirming its strong separability in the latent space.
    \item \textbf{GO:} Lowest-performing class with Precision = 0.81 and Recall = 0.84. Common misclassifications include confusion with \texttt{RIGHT} and \texttt{LEFT}.
    \item \textbf{RIGHT:} High precision (0.93) and moderate recall (0.86). Some misclassification occurs with the \texttt{GO} class.
    \item \textbf{LEFT:} Balanced performance with strong metrics across all dimensions, but minor confusion with \texttt{GO} was observed.
\end{itemize}

These results suggest that the learned latent space is semantically meaningful and well-structured, enabling accurate and efficient multi-class classification. The MLP outperformed all traditional classifiers, confirming the hypothesis that non-linear models are better suited for leveraging the representational richness of the VAE-derived latent features. This conclusion directly supports the research question RQ1 regarding the effectiveness of latent representations for downstream prediction.

\begin{itemize}
    \item \textbf{Neural Classifier and SVM} emerged as top performers, both achieving an F1-score of 0.89 and high recall for critical classes such as \texttt{STOP} and \texttt{LEFT}.
    \item \textbf{Random Forest} performed competitively, with strong recall on \texttt{STOP} (0.98), but showed minor confusion on \texttt{GO} and \texttt{LEFT}.
    \item \textbf{KNN} offered fast implementation with solid results, but is sensitive to feature scaling and suffered from overlap in ambiguous classes like \texttt{GO}.
    \item \textbf{Logistic Regression} showed the weakest performance (Accuracy = 80\%), revealing the limitations of linear separability in the VAE's latent space.
\end{itemize}

\subsection{Discussion of Classifier Results}

This comprehensive evaluation confirms that the latent space learned by the VAE is highly effective for downstream classification tasks. All classifiers achieved reasonable performance with macro F1-scores above 0.80. However, the Neural Network (MLP) classifier consistently outperformed other models in terms of both accuracy and generalization across all four driving classes. With an F1-score of 0.89 and robust recall on all classes—including the more ambiguous ones like GO and LEFT the MLP demonstrated superior capability in modeling the non-linear structure of the VAE's latent space.

While SVM and Random Forest showed competitive performance, their training flexibility and scalability were limited compared to the MLP, especially when extended to gradient-based counterfactual explanation methods. KNN and Logistic Regression served as valuable baselines but lacked the expressive power required for capturing complex class boundaries in the latent space.

Therefore, the MLP classifier was selected for all downstream experiments and counterfactual explanation tasks throughout this thesis. Its deep architecture, non-linear activation functions, and regularization mechanisms make it particularly well-suited for high-dimensional, semantically rich latent features generated by the VAE. This choice ensures consistency, interpretability, and compatibility with the explanation frameworks introduced in later chapters.

These findings empirically support the hypothesis posed in RQ1 that deep generative models like VAEs can produce compact yet semantically meaningful representations of high-dimensional driving scenes representations that can be effectively leveraged for interpretable and reliable decision-making pipelines.





\vspace{1em}

\section{Evaluation of Counterfactual Explanation Generation via Masking Techniques (RQ3)} \label{sec:masking_eval}
To address RQ3, we evaluate the quality and efficiency of counterfactual explanations generated using multiple feature masking techniques which are explained in the \cref{sec:feature_masking_pipeline}. Each method is assessed using metrics commonly adopted in counterfactual explanation literature~\cite{DELANEY2023103995, chan2022comparativestudyfaithfulnessmetrics, Singh1622975, MARKUS2021103655, DBLP:journals/corr/abs-1905-07697}, including Validity, Runtime, Sparsity, Proximity, Method Overlap, and Failure Rate. These metrics capture both the effectiveness (e.g., class change, semantic integrity, sparsity of perturbation) and efficiency (e.g., runtime, proximity) of the generated counterfactuals.
All experiments use the same VAE encoder and classifier checkpoints and are performed on identical data splits to ensure consistency and fair comparison across masking methods.
While object detection masking was initially considered, it was excluded from overlap analysis due to low detection rates and missed regions in the current dataset, resulting in poor counterfactual explanation coverage.

\subsection{Counterfactual Generation Coverage}
We first measured the number of successful counterfactuals generated by each masking method across the binary and multi-class datasets. Figures~\ref{fig:ce_count_binary} and \ref{fig:bar_chart_ce_count_multi} illustrate the total number of counterfactual explanations generated by each method, respectively.

\begin{figure}[htbp]
    \centering
    \begin{subfigure}{0.48\textwidth}
        \centering
        \includegraphics[width=\textwidth]{img/masking_results/bar_chart_explanations_2_class.png}
        \caption{CE count by method (2-class setup).}
        \label{fig:ce_count_binary}
    \end{subfigure}
    \hfill
    \begin{subfigure}{0.48\textwidth}
        \centering
        \includegraphics[width=\textwidth]{img/masking_results/bar_chart_explanations_4_class.png}
        \caption{CE count by method (4-class setup).}
        \label{fig:bar_chart_ce_count_multi}
    \end{subfigure}
    \caption[Number of counterfactuals found by method (binary vs multi-class)]{%
Comparison of the number of successful counterfactual explanations generated by different masking methods across (a) the binary-class and (b) multi-class classification setups. Each bar represents the total count of counterfactuals found using that method.}
    \label{fig:ce_count_comparison}
\end{figure}


As shown in these figures, LIME on Latent NUN outperforms other methods in binary class and grid based maskning in multi-class settings. It successfully explained 2,298 out of 2,422 images (94.88\%) in the binary setup. While 2,395 images out of 2422 images (98.88\%) in the 4-class case. In contrast, LIME on Latent feature masking using median values shows the lowest performance, with only 6.98\% and 12.26\% success in the binary and multi-class settings, respectively.

\subsection{Masking Method Overlap} \label{subsubsec:masking_method_overlap}
To understand how these methods complement one another, we analyzed the overlaps in counterfactual explanations using Venn diagrams.

\begin{figure}[htbp]
    \centering
    \begin{subfigure}{0.45\textwidth}
        \centering
        \includegraphics[width=\textwidth]{img/masking_results/venn_2_class.png}
        \caption{Method overlap (2-class).}
        \label{fig:venn_binary}
    \end{subfigure}
    \hfill
    \begin{subfigure}{0.45\textwidth}
        \centering
        \includegraphics[width=\textwidth]{img/masking_results/venn_4_class.png}
        \caption{Method overlap (4-class).}
        \label{fig:venn_multi}
    \end{subfigure}
    \caption[Overlap of counterfactuals per method using Venn diagrams]{%
Venn diagrams showing the overlap of counterfactual explanations generated by different masking methods in the (a) 2-class and (b) 4-class setups. These visualizations highlight where methods produce unique or shared explanations.}
    \label{fig:venn_comparison}
\end{figure}



From Figure~\ref{fig:venn_binary}, we observe:
\begin{itemize}
    \item 808 images (33.36\%) are explained by Grid-Based, LIME on Images, and LIME on Latent NUN masking methods together.
    \item 110 images (4.54\%) are explained by all four methods.
    \item 659 images (27.21\%) are exclusively explained by LIME on Latent NUN masking method.
\end{itemize}

Similarly, Figure~\ref{fig:venn_multi} reveals:
\begin{itemize}
    \item 1,820 images (75.14\%) are explained by Grid-Based, LIME on Images, and LIME on Latent NUN together.
    \item Only 249 images (10.28\%) are consistently explained by all four techniques.
\end{itemize}

These results indicate that while LIME on Latent feature masking contributes marginally when used alone, the LIME on Latent features masking using NUN method technique not only achieves broad coverage but also offers unique explanations not captured by other methods.


\subsection{Validity of Counterfactual Explanations}
As discussed in section~\ref{subsec:background_desirable_properties_of_CEs}, validity is a crucial metric for assessing counterfactual explanations. Table~\ref{tab:ce_validity} presents the percentage of valid counterfactuals per masking method---i.e., cases where the predicted class changed post-masking while maintaining high-quality reconstruction. In this work, a counterfactual is considered valid if it alters the model’s output to the desired target class (e.g., from \texttt{STOP} to \texttt{GO}, or from \texttt{RIGHT} to \texttt{LEFT}).

The validity percentage is computed using the following formula:

\[
\text{Validity (\%)} = \left( \frac{\text{Successful Counterfactuals}}{\text{Total Counterfactuals}} \right) \times 100.
\]

\begin{table}[htbp]
\centering
\resizebox{\textwidth}{!}{%
\begin{tabular}{lccc}
\toprule
\textbf{Method} & \textbf{Binary CE Validity (\%)} & \textbf{Multi-Class CE Validity (\%)} & \textbf{Interpretation} \\
\midrule
Grid-Based Masking & 70.52 & 98.89 & Robust and fast; excellent for multi-class \\
LIME on Latent Maksing   & 6.98  & 12.26 & Limited effect, likely due to poor latent locality \\
LIME on Image Maksing    & 38.32 & 87.65 & Effective but slower due to pixel masking \\
LIME on Latent NUN& \textbf{94.88} & \textbf{97.44} & Highest performance and generalizability \\
\bottomrule
\end{tabular}%
}
\caption{Comparison of counterfactual explanation validity across masking methods.}
\label{tab:ce_validity}
\end{table}

As seen in Table~\ref{tab:ce_validity}, the LIME on Latent NUN method achieves the highest validity, confirming its effectiveness in producing meaningful and causally influential perturbations. This supports its alignment with the theoretical definition of a "valid counterfactual" described in the background (Section~\ref{subsec:background_desirable_properties_of_CEs}). On the other hand, traditional LIME on Latent (Median) shows poor validity despite being sparse indicating that minimal changes do not necessarily result in effective counterfactuals unless they target truly causal dimensions.



\subsection{Sparsity and Proximity Analysis}

Sparsity and proximity are important dimensions of interpretability, as discussed in Section~\ref{subsec:background_desirable_properties_of_CEs}. Sparsity refers to the number of features or regions modified to achieve a prediction change, while proximity quantifies the similarity between the original and counterfactual instances, typically using the L2 distance in latent space. These two metrics jointly reflect how minimally and plausibly a counterfactual modifies the original input—a crucial consideration for trust and usability in real-world AI systems~\cite{Singh1622975}.

In this work, sparsity was defined based on the domain in which the masking was applied. For latent space methods such as LIME on Latent (Median) and LIME on Latent using NUN, sparsity was computed as the number of latent dimensions replaced during counterfactual generation. These values were logged in the output files under the “Features Replaced” column. For Grid-Based Masking, sparsity was computed as the number of image grid patches masked. These grid indices were explicitly stored in the results logs\footnote{\url{https://github.com/darshandodamani/A-Counterfactual-Explanation-Approach-Using-Deep-Generative-Models/tree/main/results/masking}} and retrieved to compute average sparsity per instance. In the case of LIME on Image, sparsity was assigned a fixed value of 128, as all superpixels selected by LIME were masked at once, representing maximum perturbation across the image space.

Proximity, on the other hand, was computed uniformly across all methods in the latent space. Specifically, it was calculated as the Euclidean (L2) distance between the original latent vector $\mathbf{z}$ and the re-encoded counterfactual latent vector $\mathbf{z'}$, as follows:

\[
\text{Proximity} = \|\mathbf{z} - \mathbf{z'}\|_2
\]

This ensures that comparisons are fair and consistent, regardless of whether the masking was applied in the image space or the latent space. Since classification occurs in the latent space, this choice of proximity metric directly reflects how much the latent semantics change as a result of the intervention.

\vspace{1em}
\begin{table}[htbp]
\centering
\small
\begin{tabular}{lcccc}
\toprule
\textbf{Method} & \textbf{Avg. Sparsity} & \textbf{Avg. Proximity} & \textbf{Coverage (\%)} & \textbf{Failure Rate (\%)} \\
\midrule
Grid-Based Masking       & 60.2    & 5.9   & 98.89 & 1.11 \\
LIME on Latent (Median)  & \textbf{2.3}     & \textbf{0.25}  & 12.26 & 87.74 \\
LIME on Latent NUN       & 75.3    & 7.4   & 97.44 & 2.56 \\
LIME on Image            & 128.0   & 25.2  & 87.65 & 12.35 \\
\bottomrule
\end{tabular}
\caption{Average sparsity and proximity of successful counterfactuals across masking methods (multi-class setup).}
\label{tab:sparsity_proximity}
\end{table}

LIME on Latent (Median) stands out as the sparsest method, requiring minimal intervention to generate counterfactuals. However, its limited effectiveness is reflected in poor CE coverage and high failure rate (Table~\ref{tab:ce_validity}), indicating that overly simplistic changes may fail to achieve meaningful prediction flips. In contrast, LIME on Latent NUN modifies more features on average, yet offers significantly better validity and class coverage, highlighting its ability to generate targeted and semantically rich counterfactuals.

As described earlier in Section~\ref{subsec:background_desirable_properties_of_CEs}, low proximity is desirable as it implies that the counterfactual remains close to the original instance, preserving realism and interpretability. From this perspective, LIME on Latent (Median) shows the best proximity score, but again, with the trade-off of low success rate. Grid-Based Masking and LIME on Latent NUN offer a better balance, providing competitive proximity while maintaining high coverage and success rates.

It is worth noting that while sparsity and proximity are reliable metrics for latent-space interventions, they are less meaningful in image space due to the nature of VAE reconstruction. For image-based methods like LIME on Image or Grid-Based Masking, masked regions can introduce artifacts or out-of-distribution distortions, which propagate through the encoder and affect proximity values. As a result, although proximity is consistently reported for all methods, its interpretability is strongest for latent-based techniques, where both the intervention and evaluation occur in a structured and learned representation space.

These differences are further discussed in Section~\ref{subsubsec:qualitative_examples}, where qualitative visualizations help clarify the implications of masking induced distortions and their influence on semantic plausibility. Overall, sparsity and proximity remain essential for comparing methods, but must be interpreted in the context of where the intervention occurs and how reconstruction artifacts might affect the underlying latent semantics.


\subsection{Per-Class Counterfactual Explanantions Success}
Table~\ref{tab:classwise_ce_multi} reports the number of valid counterfactuals generated per class.

\begin{table}[htbp]
\centering
\scriptsize
\begin{tabular}{lcccccccc}
\toprule
\textbf{Method} & \textbf{STOP (\%)} & \textbf{GO (\%)} & \textbf{LEFT (\%)} & \textbf{RIGHT (\%)} & \textbf{Total CE Found} & \textbf{Total CE (\%)} & \textbf{Total Time (min)} \\
\midrule
Grid-Based        & 100.0 & 100.0 & 89.0  & 100.0 & 2395 & 98.89 & 12.98 \\
LIME on Latent    & 10.3  & 9.4   & 18.4  & 26.1  & 297  & 12.26 & 92.63 \\
LIME on Image     & 100.0 & 100.0 & 1.6   & 74.5  & 2123 & 87.65 & 78.65 \\
LIME on Latent NUN & 98.6  & 96.0  & 97.2  & 99.6  & \textbf{2360} & \textbf{97.44} & 263.79 \\
\bottomrule
\end{tabular}
\caption{Per-class counterfactual explanation success (multi-class setup).}
\label{tab:classwise_ce_multi}
\end{table}

The table shows that while most methods effectively capture STOP and GO transitions, LIME on Latent NUN exhibits the most balanced performance across all classes, closely followed by the Grid-Based method.




\vspace{1em}


\subsection{Efficiency: Time Complexity and Execution Time}
As discussed in the Background (see Section~\ref{subsec:background_desirable_properties_of_CEs}), one of the desirable properties of a counterfactual explainer is \textit{efficiency} the ability to generate explanations within a reasonable time, particularly in real-time or interactive systems like autonomous driving. In this evaluation (Section~\ref{sec:masking_eval}), we observe significant variation in execution times across the proposed methods.

While the LIME on Latent NUN approach consistently produced the highest-quality and most valid counterfactuals, it was also the most computationally expensive, requiring over 263 minutes to process the entire dataset. In contrast, Grid-Based Masking achieved nearly equivalent performance in terms of coverage and validity but completed in just 13 minutes. This highlights the trade-off between explanation quality and computational efficiency—an important consideration when selecting methods for deployment in time-sensitive applications.



\subsection{Qualitative Examples of Counterfactual Explanations} \label{subsubsec:qualitative_examples}
To complement the quantitative evaluation of counterfactual explanation using different masking methods, qualitative visual analyses were conducted to assess the visual plausibility, semantic preservation, and discriminative alterations introduced by each masking technique. These visual examples help contextualize how each method perturbs the input to find counterfactual explanations that lead to prediction changes.

Individual counterfactual examples for all five masking methods Grid-Based Masking, Object Detection-Based Masking, LIME on Image, LIME on Latent Features, and LIME on Latent using NUN are provided in \cref{sec:feature_masking_pipeline} (see Figures~\ref{fig:grid_ce_example}--\ref{fig:object_detection_masking}). Each figure narratively presents:

\begin{itemize}
    \item The original input image and its predicted class,
    \item The type of masking applied (e.g., grid, object, latent),
    \item The reconstructed image after masking via the VAE,
    \item The change in prediction resulting from the perturbation.
\end{itemize}

This structure highlights the causally relevant regions identified by each method and their impact on the classification outcome.

Each method is qualitatively evaluated based on:
\begin{itemize}
    \item Semantic fidelity: Whether the essential image structure is preserved after masking and reconstruction.
    \item Visual realism: The plausibility of the counterfactual image.
    \item Effectiveness: Whether the counterfactual successfully changes the predicted class.
\end{itemize}

\vspace{0.5em}

To consolidate these comparisons, Table~\ref{tab:cf_visual_examples} provides an overview of one representative counterfactual explanation per method. These samples were selected from the test set where class flips occurred, illustrating how each masking technique alters the input to change the classifier's decision.

\begin{table}[htbp]
    \centering
    \caption[Counterfactual examples across masking methods]{%
Representative counterfactual explanations across different masking methods. In each case, the original image is labeled \textbf{STOP}, and the counterfactual image is labeled \textbf{GO} due to the masking or removal of the STOP sign or related features.}
    \label{tab:cf_visual_examples}
    \begin{tabular}{>{\centering\arraybackslash}p{3.5cm} >{\centering\arraybackslash}c >{\centering\arraybackslash}c}
        \toprule
        \textbf{Masking Method} & \textbf{Original Image (STOP)} & \textbf{Counterfactual (GO)} \\
        \midrule
        Grid-Based Masking & 
        \includegraphics[width=0.3\textwidth]{img/masking_results/original.png} & 
        \includegraphics[width=0.3\textwidth]{img/masking_results/grid_cf.png} \\
        \addlinespace
        Object Detection Masking & 
        \includegraphics[width=0.3\textwidth]{img/masking_results/original.png} & 
        \includegraphics[width=0.3\textwidth]{img/masking_results/object_detection_cf.png} \\
        \addlinespace
        LIME on Image & 
        \includegraphics[width=0.3\textwidth]{img/masking_results/original.png} & 
        \includegraphics[width=0.3\textwidth]{img/masking_results/lime_image_cf.png} \\
        \addlinespace
        LIME on Latent & 
        \includegraphics[width=0.3\textwidth]{img/masking_results/original.png} & 
        \includegraphics[width=0.3\textwidth]{img/masking_results/lime_latent_cf.png} \\
        \addlinespace
        LIME on Latent NUN & 
        \includegraphics[width=0.3\textwidth]{img/masking_results/original.png} & 
        \includegraphics[width=0.3\textwidth]{img/masking_results/lime_NUN_cf.png} \\
        \bottomrule
    \end{tabular}
\end{table}


\textbf{Observations:}
\begin{itemize}
    \item \textbf{Grid-Based Masking:} Often introduces localized changes with some visible artifacts but successfully flips the class in most cases. It demonstrates good semantic targeting, although the visual naturalness may sometimes be compromised.
    \item \textbf{Object Detection Masking:} Attempts to remove salient objects. However, due to limitations in YOLOv5's detection performance on this dataset, many regions are missed. This leads to lower counterfactual success rates and less informative counterfactuals. Object detection masking is retained here primarily for completeness.
    \item \textbf{LIME on Image:} Produces counterfactuals that are visually intuitive by masking semantically important regions (e.g., road signs, motion cues). Although the reconstructions are perceptually realistic, the region selection can sometimes lack subtlety.
    \item \textbf{LIME on Latent:} Directly alters internal representations, often resulting in blurry or implausible reconstructions. This can lead to noisy or confusing counterfactuals that fail to reliably flip the class.
    \item \textbf{LIME on Latent NUN:} Achieves the best balance by minimally perturbing only the most relevant latent features. Its counterfactuals are both semantically meaningful and visually realistic, leading to successful class flips with minimal distortion.
\end{itemize}

\vspace{1em}

Together, these qualitative comparisons reinforce the quantitative results presented earlier (Figures~\ref{fig:bar_chart_ce_count_multi}--\ref{fig:venn_comparison}). In particular, LIME on Latent based masking using NUN method consistently generates the most faithful and effective counterfactual explanations. The visual clarity and semantic integrity of its outputs further confirm its practical utility for explainability in autonomous driving models.


\subsection{Discussions on Counterfactual Explanation Generation via Masking Techniques}

The experimental results provide a comprehensive comparison of the five masking-based counterfactual generation techniques and highlight clear trends in their relative strengths and limitations. Among all evaluated methods, LIME-guided latent feature masking using Nearest Unlike Neighbor (NUN) emerged as the most effective approach. It consistently produced high-quality counterfactuals that were both class-discriminative and semantically aligned with the original input, demonstrating superior performance across nearly all evaluation metrics. However, this method comes with a notable trade-off—its computational cost. With an average runtime exceeding 263 minutes for the full dataset, it poses challenges for deployment in time-sensitive or real-time environments.

Grid-Based Masking, on the other hand, offered a compelling alternative. While slightly less effective in terms of semantic richness and interpretability, it excelled in coverage, speed, and visual coherence. With an execution time of just 13 minutes and near-perfect class-wise counterfactual success, it serves as a strong baseline that balances performance and practicality, especially for multi-class prediction tasks in autonomous driving.

In contrast, LIME-based masking on latent features (with median replacement) exhibited significant limitations. Although it achieved the lowest average sparsity changing as few as 2.3 latent dimensions per counterfactual—it suffered from poor validity and limited coverage. This suggests that minimal changes alone are not sufficient if they fail to meaningfully influence the classifier's decision boundary. Similarly, LIME on Image, while more successful in terms of coverage, produced counterfactuals with very high sparsity (treated as 128), resulting in distorted reconstructions and poor interpretability. The primary issue here stems from the masking strategy itself—zeroing out critical image regions leads to unnatural artifacts that the VAE, not trained on such distributions, cannot realistically reconstruct. This severely limits its practical value, particularly for scenes requiring subtle visual coherence.

Object Detection-based Masking performed the worst overall. The main bottleneck was the failure of YOLOv5 to consistently detect relevant objects in the CARLA dataset, leading to unreliable or incomplete masking. As a result, this method frequently failed to generate valid counterfactuals and was excluded from certain evaluations due to insufficient sample coverage.

From a minimality perspective, LIME on Latent (Median) remains the most sparse technique but falls short in altering the classifier’s prediction. LIME on Image, by contrast, modifies the entire image with excessive masking, compromising realism. Grid-Based and LIME on Latent NUN methods demonstrate the most promising trade-offs, achieving a favorable balance between sparsity, proximity, validity, and execution time. These findings reinforce that counterfactual explanations must not only flip the model's prediction but do so through changes that are visually plausible, semantically meaningful, and computationally viable.

Taken together, the results strongly advocate for latent-space based masking techniques particularly LIME on Latent using NUN as the most interpretable and effective strategy for counterfactual generation in autonomous driving settings as shown in this thesis. While further optimization is required to reduce its computational overhead, this approach holds considerable promise for enhancing transparency and trust in real-world decision-making systems.





\section{Human-Centered Evaluation: Methodology and Results (RQ4)} \label{sec:human_evaluation}

While traditional quantitative metrics such as SSIM and PSNR offer insight into the visual quality of reconstructed counterfactual images, they do not fully reflect human judgment regarding interpretability, plausibility, or realism. As noted by Delaney et al.~\cite{DELANEY2023103995}, counterfactual explanations are meant to fulfill human explanation goals and must therefore be evaluated with human-centered metrics. This is especially crucial in high-stakes domains such as autonomous driving, where explanations must not only be technically valid but also intuitively understandable and semantically coherent from a user's perspective.

To incorporate this human-centered evaluation, we developed a web-based application using FastAPI for backend processing and Jinja2 for rendering the frontend. The system was designed to present users with the original input image, its predicted label, and four corresponding counterfactual explanations each generated using a different masking technique: Grid-Based Masking, LIME on Images, LIME on Latent Features, and LIME on Latent Features using the NUN method. To ensure fairness and avoid bias, the interface randomized the order of counterfactuals and concealed the identity of the underlying methods. Participants were only shown the prediction labels associated with each counterfactual image, not the technique used to generate it. A screenshot of the evaluation interface is provided in Appendix~\ref{appendix:webinterface}, Figure~\ref{fig:app:form_ui}.

Object detection-based masking was excluded from this user study due to its low counterfactual success rate. It failed to consistently produce valid counterfactual examples and thus lacked sufficient representative samples for human evaluation.

Each user was asked to evaluate the counterfactual explanations based on three criteria: interpretability (how clearly the image highlights the minimal change required to alter the model’s prediction), plausibility (how realistic and contextually appropriate the modified image appears), and visual coherence (whether the transformation affects only necessary regions while preserving the rest of the scene). These aspects were rated on a Likert scale of 1 to 5.\footnote{The Likert scale is a common psychometric scale used in questionnaires, where 1 typically represents "strongly disagree" or "poor" and 5 represents "strongly agree" or "excellent".} Users also had the option to provide qualitative comments to elaborate on their assessments. The samples shown were those where all four methods successfully generated counterfactuals (see Section~\ref{subsubsec:masking_method_overlap} for the selection criteria).

\subsection{Quantitative Analysis of User Ratings} \label{subsubsec:quantitative_analysis_of_user_ratings}

To address RQ4 (Which counterfactual explanation method is preferred by users when selecting among generated explanations of the same original image, and what factors influence user preference?), we conducted an in-depth analysis of the collected human ratings. The anonymized evaluation labels were internally mapped to the respective methods: Counterfactual\_1 corresponds to Grid-Based Masking, Counterfactual\_2 to LIME on Image Masking, Counterfactual\_3 to LIME on Latent Features, and Counterfactual\_4 to LIME on Latent Features using the NUN method.

The results of the user study, visualized in Figure~\ref{fig:bar_plot_user_eval}, show that Grid-Based Masking (Method 1) achieved the highest scores in both interpretability (4.32) and visual coherence (4.16), indicating its strength in generating localized, structured changes while preserving background consistency. In terms of plausibility, LIME on Latent Features (Method 3) and LIME on Latent Features using NUN (Method 4) performed best, scoring 4.04 and 3.96 respectively—reflecting their ability to generate semantically plausible modifications in the latent space. In contrast, LIME on Image Masking (Method 2) consistently scored the lowest across all criteria (1.00), with participants noting issues such as severe artifacts, unrealistic reconstructions, and lack of clarity.

\begin{figure}[htbp]
    \centering
    \includegraphics[width=0.75\textwidth]{img/human_rating_results/bar_plot_user_evaluations.png}
    \caption[Bar plot of average user ratings for CE methods]{%
Average user ratings (scale 1–5) for each counterfactual explanation method across all evaluation criteria. Higher values indicate better perceived interpretability, plausibility, or coherence.}
    \label{fig:bar_plot_user_eval}
\end{figure}


To provide a consolidated view, we generated a heatmap (Figure~\ref{fig:heatmap_user_eval}) summarizing the average user ratings per method for each evaluation criterion. This visual representation clearly demonstrates the complementary strengths of different methods. Grid-Based Masking excels in interpretability and coherence, while latent space methods, particularly Method 3 and Method 4, produce more realistic and semantically aligned modifications.

\begin{figure}[htbp]
    \centering
    \includegraphics[width=0.75\textwidth]{img/human_rating_results/heatmap_user_evaluations.png}
    \caption[Heatmap of user evaluations by method and criterion]{%
Heatmap of average user ratings (scale 1–5) for each counterfactual method across individual evaluation dimensions: interpretability, plausibility, and visual coherence. Darker shades represent stronger user agreement.}
    \label{fig:heatmap_user_eval}
\end{figure}


\vspace{0.5em}
\paragraph{Per-Criterion Analysis.}
Figures~\ref{fig:cf_interpretability}--\ref{fig:cf_visualcoherence} provide detailed views of average ratings per evaluation criterion:

\begin{itemize}
    \item Interpretability: Grid-Based Masking (Method 1) received the highest rating (4.32), followed by LIME on Latent NUN (2.64) and LIME on Latent (2.44). LIME on Image (1.00) was rated lowest due to aggressive, uninformative masking patterns.
    \item Plausibility: LIME on Latent (Method 3) was rated most plausible (4.04), with NUN-enhanced latent manipulation (Method 4) close behind (3.96). Grid-Based Masking scored moderately (2.28), while LIME on Image again received the lowest rating (1.00).
    \item Visual Coherence: Grid-Based Masking again led with 4.16, confirming its strength in preserving background information. Method 4 followed with 3.04, Method 3 scored 2.76, and Method 2 trailed at 1.00.
\end{itemize}

\begin{figure}[h]
    \centering
    \includegraphics[width=0.6\textwidth]{img/human_rating_results/Interpretability_ratings.png}
    \caption[User-rated interpretability scores by method]{%
Average user ratings for interpretability across counterfactual explanation methods (scale 1–5). Higher scores indicate better user understanding of why the prediction changed.}
    \label{fig:cf_interpretability}
\end{figure}

\begin{figure}[h]
    \centering
    \includegraphics[width=0.6\textwidth]{img/human_rating_results/Plausibility_ratings.png}
    \caption[User-rated plausibility scores by method]{%
Average user ratings for plausibility across counterfactual explanation methods (scale 1–5). Ratings reflect how realistic the generated counterfactual images appear to human observers.}
    \label{fig:cf_plausibility}
\end{figure}

\begin{figure}[h]
    \centering
    \includegraphics[width=0.6\textwidth]{img/human_rating_results/VisualCoherence_ratings.png}
    \caption[User-rated visual coherence scores by method]{%
Average user ratings for visual coherence across counterfactual explanation methods (scale 1–5). This criterion reflects image clarity, absence of artifacts, and semantic consistency with the original.}
    \label{fig:cf_visualcoherence}
\end{figure}

\vspace{0.5em}
\subsection{Qualitative Feedback and Discussion}

In addition to numerical ratings, users provided open-ended feedback to elaborate on their experience evaluating the counterfactual explanations. A thematic analysis of these comments revealed distinct strengths and weaknesses for each method.

Grid-Based Masking (Method 1) was frequently praised for its clarity and precision. Participants described it using phrases such as "clear and coherent masking", "very clear and detailed", and "excellent clarity and structure". These remarks align with its high interpretability and visual coherence scores. Feedback such as "satisfactory masking overall" and "consistent performance" further emphasize its reliability.

LIME on Image Masking (Method 2) received universally negative feedback. Users described outputs as "image mostly black", "completely dark image", "no visible detail", and "output unusable". These comments align with its lowest ratings across all evaluation dimensions, indicating that the zero-masking strategy significantly degraded image quality and interpretability.

LIME on Latent Features (Method 3) received mixed responses. Some users described the reconstructions as "moderate quality" and "acceptable", while others noted "average quality with slight artifacts" or "over-modified regions". These qualitative impressions reflect the middle-range scores of this method.

LIME on Latent Features using NUN (Method 4) was generally well-received. Comments included "realistic modification", "more details preserved", and "mimics original scene well". These remarks highlight its effectiveness in generating semantically aligned counterfactuals. However, a few responses indicated that further refinement could improve local consistency.

Overall, qualitative feedback closely mirrored the quantitative trends. Grid-Based Masking was preferred for clarity and structure, while latent space methods especially with NUN guidance—offered realism and plausibility. LIME on Image consistently underperformed. This underscores the value of integrating user feedback in evaluating counterfactual methods, especially in critical applications like autonomous driving.

\chapter{Related work} \label{Related work}
This chapter provides a comprehensive review of prior research relevant to this thesis, structured across multiple areas including explainability methods, counterfactual explanations, generative modeling approaches, semantic and spatial interventions in latent/image space, and applications in safety-critical domains. The review ends with a comparative discussion positioning this thesis in the context of existing work.



\section{Explainability in Machine Learning}
Numerous post-hoc interpretability techniques have been proposed to make complex black-box models more interpretable. Model-agnostic methods are the most widely used among them, such as LIME~\cite{Ribeiro2018} and SHAP~\cite{lundberg2017unifiedapproachinterpretingmodel}. Such techniques explain specific predictions by fitting simpler surrogate models locally around a given input. SHAP is based on cooperative game theory principles, injecting a certain amount of rigor into the process of attributing the contributions of features to the outputs while guaranteeing consistency and local accuracy.

Gradient-based approaches such as Integrated Gradients~\cite{8237336} and Influence Functions~\cite{pmlr-v70-koh17a} use model derivatives to attribute importance to input features. These methods have been widely applied in image classification, text classification, and tabular data settings. While these techniques offer insights into feature attribution, they primarily identify important features rather than provide actionable guidance on how to change a prediction.


\section{Counterfactual Explanations}
Counterfactual explanations identify minimal changes to input features that would alter a model's prediction~\cite{wachter2018CE}. They are valuable in high-stakes settings where understanding model behavior and suggesting actionable changes are essential.

In structured data domains, CEILS~\cite{crupi2021counterfactualexplanationsinterventionslatent} performs latent space interventions using structural causal models. Pawelczyk et al.\cite{Pawelczyk_2020} propose density-aware counterfactuals that maintain plausibility by optimizing within the data manifold. Ustun et al.\cite{ustun2019actionable} focus on actionable recourse, while Mothilal et al.\cite{DBLP:journals/corr/abs-1905-07697} introduce diversity to generate multiple plausible counterfactuals. Karimi et al.\cite{karimi2020algorithmic} emphasize the integration of causal assumptions.

Recent optimization-based methods balance constraints such as proximity, sparsity, plausibility, and diversity~\cite{NEURIPS2021_fd0a5a5e, delser2022tradeoff}, but their extension to unstructured data like images is non-trivial due to dimensionality and perceptual realism requirements.




\section{Generative Models for Counterfactual Explanations}

To address the challenge of generating counterfactuals for high-dimensional inputs such as images, several studies have adopted generative models like Variational Autoencoders (VAEs) and Generative Adversarial Networks (GANs).

VAEs~\cite{Kingma_2019} encode data into a structured latent space that facilitates controlled generation of new samples. Enhancements like log-cosh loss~\cite{chen2019log} improve reconstruction robustness. Ernst~\cite{ernst2024counterfactual} modifies the VAE objective for anomaly detection and interpretable counterfactuals in tabular domains.

The Contrastive Explanation Method (CEM)~\cite{DBLP:journals/corr/abs-1802-07623} introduces pertinent positives and negatives using a convolutional autoencoder to generate contrastive explanations. SharpShooter~\cite{barr2021counterfactualexplanationslatentspace} interpolates between latent encodings of different classes to induce classifier prediction changes. GAN-based techniques, such as Residual GANs~\cite{nemirovsky2021countergangeneratingrealisticcounterfactuals}, add adversarial perturbations to the latent space for generating class-altering counterfactuals.


\section{Semantic and Spatial Masking Techniques for Visual Counterfactuals}
Another approach to generating interpretable visual counterfactuals involves modifying semantically meaningful image regions. This includes masking detected objects or spatially-defined regions, then reconstructing them to observe changes in classification.

OCTET~\cite{zemni2023octetobjectawarecounterfactualexplanations} modifies objects in autonomous driving scenes to flip predictions, preserving semantic coherence and realism. Grid-based and saliency-based masking provide localized focus, though may lack semantic granularity.

Agarwal and Nguyen~\cite{agarwal2020explainingimageclassifiersremoving} propose combining attribution-based masks with pretrained inpainting models like DeepFill, improving visual plausibility and reducing interpretive artifacts.

\section{Attribution-Guided Counterfactual Generation}
Recent works integrate attribution methods into the counterfactual generation process to identify impactful features for alteration.

Wijekoon et al.~\cite{WijekoonWNMPC21} propose the Nearest Unlike Neighbor (NUN) strategy, using LIME to rank features and sequentially replacing them based on importance. This results in sparse and interpretable counterfactuals.

DisCERN~\cite{wiratunga2021discerndiscoveringcounterfactualexplanationsusing} uses feature relevance from LIME or SHAP to guide counterfactual edits, avoiding optimization-based adaptation. A variant applies Integrated Gradients with class-specific baselines, reducing feature modifications and improving explanation stability.

Hybrid methods that combine attribution and latent space editing (e.g., with VAEs or GANs) offer greater semantic control and visual consistency, particularly in high-dimensional image domains.

\section{Applications of Counterfactual Explanations}
Counterfactual explanations have been applied in various domains where interpretability and fairness are critical. In finance, they aid in loan decisions and credit scoring~\cite{guidotti2022counterfactual, DELANEY2023103995}. In healthcare, they highlight changes in clinical features affecting diagnoses~\cite{10.1145/3351095.3372855}. In education and employment, they explain student and hiring outcomes~\cite{WijekoonWNMPC21}.

In autonomous driving, counterfactuals help debug perception modules. Zemni et al.\cite{zemni2023octetobjectawarecounterfactualexplanations} demonstrate modifying scene elements to correct misclassifications. Rudin\cite{Rudin2019} emphasizes the use of inherently interpretable models in life-critical applications, reinforcing the importance of reliable visual explanations.

\section{Summary and Positioning of This Work}
This thesis builds on prior research in generative modeling, attribution-guided interventions, and semantic masking to develop a VAE-based counterfactual explanation framework tailored for autonomous driving.

Unlike interpolation-based or adversarial techniques, the implemented approach in this thesis focuses on semantically grounded masking strategies including image based, and latent feature masking to flip classifier predictions while preserving realism. LIME-based attribution and Nearest Unlike Neighbor (NUN) guidance are integrated in the latent space to enhance sparsity and interpretability.

Evaluation includes both quantitative metrics and a user study, aligning with recent research emphasizing human-centered assessment~\cite{DELANEY2023103995}. This positions the work within the broader goal of building trustworthy, interpretable AI systems for safety-critical applications.

\chapter{Conclusion and Future Work} \label{Conclusion and Future Work}
In this work, we have laid the groundwork for a process of rigorously using suitable masking technique to generate the counterfactual explanation in the driving task. 




Several avenues for future work open up. First, all existing methods make recommendations of how features would need to be altered to receive a desired result, but none of these methods give associated input importance. And second, it would be desirable to formalize the tradeoff between the variational autoencoder capacity and counterfactual faithfulness.




Key findings from AI-based evaluation and human study were compared.
Which masking method performed best overall?
Are AI-based metrics reliable indicators of human preference?
Limitations and future improvements were discussed.


% Bibliography
\printbibliography[heading=bibintoc]

% Appendix
% \appendix
% \chapter{My First Appendix}
% \chapter{Definitions} \label{Definitions}
% \input{text/Definitions.tex}